%!TEX root = ../thesis.tex
\chapter{Abstract}
\small
Differential equations are used ubiquitously to predict and understand phenomena across many fields.
These models are inevitably subject to uncertainties arising from a range of sources, including measurement error, unresolved components, and resolution limitations.
Accounting for these uncertainties leads to better models, but this is analytically challenging and often requires computationally expensive Monte Carlo simulation.
In this thesis, we consider It\^o stochastic differential equations, which extend ordinary differential equations to include stochastic terms and provide a rich framework for explicitly parameterising uncertainties in the model.
We look to address the need for computationally efficient methods that characterise uncertainty without resorting to the alternative of bulk simulation.

We first build upon previous small-noise studies to provide an explicit bound for the error between stochastic differential equations and corresponding linearisations written in terms of a deterministic system.
Our framework accounts for non-autonomous coefficients, multiplicative noise, and uncertain initial conditions.
These linearisations are solvable and efficient to compute and so can serve as an approximate solution to the stochastic differential equation.
We demonstrate the predictive power of our bound on several toy examples, providing for the first time a numerical validation of linearisation approximations of stochastic differential equations.
In characterising this relationship, we are also able to extend stochastic sensitivity (Balasuriya, \emph{SIAM Review}, 2020:781-816), a recently introduced tool for characterising the impact of uncertainty on differential equation solutions.
Stochastic sensitivity was previously restricted to 2-dimensional flows and we overcome this limitation to empower the use of these tools on models of arbitrary dimension.

We also propose an \emph{ad hoc} algorithm for approximating a stochastic differential equation solution with a Gaussian mixture model constructed from many different linearisations.
Such an algorithm is computationally efficient and provides an analytic probability density function, unlike stochastic samples.
Critically, the algorithm can capture non-Gaussian features in the stochastic solution that a single linearisation cannot.
Our initial investigation into this algorithm, using a data-driven model of a drifter in the Gulf Stream, yielded promising results to motivate future development.

Our work provides many avenues for further development, some of which we discuss briefly.
This includes theoretical extension to a broader range of stochastic models driven by different types of noise and implications for the Fokker-Planck equation, an equivalent representation of a stochastic differential equation.
We also anticipate applications within the fields of data assimilation, stochastic parameterisation, and, through a connection with results for discrete stochastic systems, mathematical epidemiology.
