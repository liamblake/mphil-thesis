\chapter{Details of population process models}\label{app:epi}
In this appendix, we supplement the discussion on population processes and continuous time Markov chains in \Cref{sec:epi_disc} by providing further details on the theoretical results by \citet{Kurtz_1970_SolutionsOrdinaryDifferential,Kurtz_1971_LimitTheoremsSequences}, the method for simulating from CTMCs, and the details of the 5-dimensional Ebola model by \citet{LegrandEtAl_2007_UnderstandingDynamicsEbola}.

\section{Coefficients of the fluid and diffusion limits}\label{app:epi_kurtz}
Let \(X_t\) denote the unscaled \(n\)-dimensional continuous-time Markov chain with state space \(\mathcal{S} \subseteq \set{0, 1, \dotsc, M}^n\), so that the process is evolving with a fixed population size \(M\).
Let \(q\colon \mathcal{S}\times\mathcal{S} \to [0,\infty)\) denote the transition rates between two states of the Markov chain
The process \(X_t\) is termed \emph{density dependent} (in the sense of \citet{Kurtz_1970_SolutionsOrdinaryDifferential}) if
\begin{equation}
	q\!\left(n, n+l\right) = Mf\!\left(\frac{n}{M}, l\right),
	\label{eqn:ctmc_dens_dep}
\end{equation}
where \(f\) is a suitable function and \(n, n+l \in \mathcal{S}\).
% Typically, \(N\) is related to the size of the system and \(n / N\) is a population density.
% \section{Overview of the fluid and diffusion limits}\label{app:epi_diffusion}
% A population process is \emph{density dependent} (in the sense of \citet{Kurtz_1970_SolutionsOrdinaryDifferential}) if there is a parameter \(N\) such that
% \begin{equation}
% 	q\!\left(n, n+l\right) = Nf\!\left(\frac{n}{N}, l\right),
% 	\label{eqn:ctmc_dens_dep}
% \end{equation}
% where \(f\) is a suitable function and \(n, n+l \in \mathcal{S}\).
% Typically, \(N\) is related to the size of the system and \(n / N\) is a population density.
% The condition in \cref{eqn:ctmc_dens_dep} states that the transition rates of the process \(X_t\) depends on \(X_t\) itself only through the density \(X_t / N\).
Let \(Y_t^{(N)}\) describe the \(n\)-dimensional density process, i.e.\ with \(i\)th element \(X_t^{(i)} / M\).
Theorem 3.1 of \citet{Kurtz_1970_SolutionsOrdinaryDifferential} establishes that in the large population limit \(M \to \infty\), the density process \(Y_t^{(N)}\) converges in probability to a deterministic trajectory \(Y_t^{(\infty)}\) solving the ODE
\begin{equation}
	\dod{Y_t^{(\infty)}}{t} = Q\!\left(Y_t^{(\infty)}\right), \quad Y_0^{(\infty)} = X_0 / M,
	\label{eqn:ctmc_dens_dep_ode}
\end{equation}
where
\[
	Q\!\left(y\right) = \sum_{\substack{l \in \R^n \\ l \neq 0,\, y + l \in \mathcal{S}}}{l f\!\left(y, l\right)}.
\]
For large \(M\), the density process has small variation and is ``close'' to the deterministic solution to \cref{eqn:ctmc_dens_dep_ode}.
% This result is analogous to that for small noise SDEs; the solution to a stochastic differential equation formally converges to that of a deterministic system (involving only the drift term) in the limit of small noise, which is a result established by large deviations principles \citep[e.g]{FreidlinWentzell_1998_RandomPerturbationsDynamical}.
\citet{Kurtz_1971_LimitTheoremsSequences} then established a stronger result, showing that the variation of the density process about this deterministic limit is captured by an It\^o diffusion.
Define the scaled process
\[
	Z_t^{(M)} = \sqrt{M}\left(Y_t^{(M)} - Y_{t}^{(\infty)}\right),
\]
then Theorem 3.5 of \citet{Kurtz_1971_LimitTheoremsSequences} proves that \(Z_t^{(M)}\) converges in distribution (weakly) to an It\^o diffusion \(Z_t^{(\infty)}\) solving
\begin{equation}
	\dif Z_t^{(\infty)} = \nabla Q\!\left(Y_t^{(\infty)}\right) Z_t^{(\infty)}\dif t + G\!\left(Y_t^{(\infty)}\right)\dif W_t, \quad Z_0^{(\infty)} = 0,
	\label{eqn:ctmc_dens_dep_sde}
\end{equation}
where \(W_t\) is an \(n\)-dimensional Wiener process and the \(n \times n\) diffusion matrix \(G\) is such that
\begin{equation}
	\left[G\!\left(y\right)G\!\left(y\right)^{\T}\right]_{ij} = \sum_{\substack{l \in \R^n \\ l \neq 0,\, y + l \in \mathcal{S}}}{l_i l_j f\!\left(y,l\right)}
	\label{eqn:ctmc_dens_dep_sde_diff_cond}
\end{equation}
Any choice of diffusion matrix \(G\) such that \cref{eqn:ctmc_dens_dep_sde_diff_cond} is satisfied will result in a statistically identical stochastic process \(Z_0^{(\infty)}\).


\section{Simulating from a continuous-time Markov chain}\label{app:ctmc_sim}
A single sample \(X\) from the population process at time \(T\) can be simulated with the following procedure \citep{Gillespie_1977_ExactStochasticSimulation}:
\begin{enumerate}
	\item Initialise \(X = X_0\), the initial state of the process.
	      If the initial state is uncertain, sample \(X\) from the initial state distribution.

	\item Sample the time \(\tau\) to the next event as
	      \[
		      \tau \isExp{\sum_{\substack{l \in \R^n \\ l \neq 0,\, y + l \in \mathcal{S}}}{q\!\left(X, X + l\right)}},
	      \]
	      and set \(t = t + \tau\).

	\item If \(t > T\), terminate.
	      Otherwise, sample the next state from the set of possible transitions \(\setc{l \in \R^n}{l \neq 0,\, y + l \in \mathcal{S}}\), where the probability of transitioning from \(X\) to \(X + l\) is given by
	      \[
		      \frac{q\!\left(X, X + l\right)}{\sum_{\substack{k \in \R^n \\ k \neq 0,\, y + k \in \mathcal{S}}}{q\!\left(X, X + k\right)}}.
	      \]
	      Set \(X\) to this state.

	\item Repeat Steps 2 and 3 until the sample path is terminated.

\end{enumerate}
The result is a single sample \(X\) of the population process at time \(T\).
The sum is taken across all the possible transitions from the current state.
Note that typically most of these rates are zero, so the sum does not need to be taken over the entire state space.


\section{Details of 5-dimensional Ebola model}\label{app:ebola}
	\begin{figure}
		\begin{center}
			\begin{tabular}{|p{5cm}|c|p{5cm}|}
				\hline
				Parameter                                                               & Value               & Source                                                                                                                                 \\ \hhline{|=|=|=|}
				Population size \(M\)                                                   & 200000              & \citet{DowellEtAl_1999_TransmissionEbolaHemorrhagic}                                                                                   \\ \hline
				Initial cases \(I_0\)                                                   & 3                   & \citet{KhanEtAl_1999_ReemergenceEbolaHemorrhagic}                                                                                      \\ \hline
				Mean duration of incubation	\(\alpha\)                                  & \(1 / 7\) per day   & \citet{DowellEtAl_1999_TransmissionEbolaHemorrhagic,BwakaEtAl_1999_EbolaHemorrhagicFever,NdambiEtAl_1999_EpidemiologicClinicalAspects} \\ \hline
				Mean time to hospitalisation \(\gamma_h\)                               & \(1 / 9.6\) per day & \citet{KhanEtAl_1999_ReemergenceEbolaHemorrhagic}                                                                                      \\ \hline
				Mean duration of infectious period \(\gamma_i\)                         & \(1 / 10\) per day  & \citet{DowellEtAl_1999_TransmissionEbolaHemorrhagic,RoweEtAl_1999_ClinicalVirologicImmunologic}                                        \\
				Mean time to death \(\gamma_d\)                                         & \(1 / 2\) per day   & \citet{KhanEtAl_1999_ReemergenceEbolaHemorrhagic}                                                                                      \\ \hline
				Proportion of hospitalised cases \(\theta_1\)                           & \(0.67\)            & \citet{KhanEtAl_1999_ReemergenceEbolaHemorrhagic}                                                                                      \\ \hline
				Unhospitalised Case-fatality ratio (with burial) \(\delta_1\)           & \(0.8\)             & \citet{KhanEtAl_1999_ReemergenceEbolaHemorrhagic}                                                                                      \\ \hline
				Hospitalised case-fatality ratio (with burial)\(\delta_2\)              & \(0.8\)             & \citet{KhanEtAl_1999_ReemergenceEbolaHemorrhagic}                                                                                      \\ \hline
				Transmission rate in the community \(\beta_I\)                          & \(0.588\) per week  & \citet{LegrandEtAl_2007_UnderstandingDynamicsEbola}                                                                                    \\ \hline
				Transmission rate at the hospital \(\beta_H\)                           & \(0.794\) per week  & \citet{LegrandEtAl_2007_UnderstandingDynamicsEbola}                                                                                    \\ \hline
				Transmission rate during funerals \(\beta_F\)                           & \(7.653\) per week  & \citet{LegrandEtAl_2007_UnderstandingDynamicsEbola}                                                                                    \\ \hline
				Mean time from hospitalisation to death \(\gamma_{dh}\)                 & per week            & \citet{LegrandEtAl_2007_UnderstandingDynamicsEbola}                                                                                    \\ \hline
				Mean time from hospitalisation to end of infectiousness \(\gamma_{ih}\) & per week            & \citet{LegrandEtAl_2007_UnderstandingDynamicsEbola}                                                                                    \\ \hline
			\end{tabular}
		\end{center}
		\caption{Parameter values for the Ebola model, estimated from the 1995 outbreak in the Democratic Republic of Congo.
			This table is adapted from Tables 3 and 4 of \citet{LegrandEtAl_2007_UnderstandingDynamicsEbola}, where values sourced directly from \citet{LegrandEtAl_2007_UnderstandingDynamicsEbola} have been estimated from morbidity data.}
		\td{Descriptions of variables not quite right}
		\label{tab:ebola_param_vals}
	\end{figure}

To conclude the discussion on population processes in \Cref{sec:epi_disc}, we considered the 5-dimensional model by \citep{LegrandEtAl_2007_UnderstandingDynamicsEbola} of the 1995 outbreak of Ebola in the Democratic
In this appendix, we provide further details on the model.
\Cref{tab:ebola_param_vals} provides parameter values for the model, including the physical interpretations of each parameter.
These parameter values are taken from a combination of previous studies \citep{DowellEtAl_1999_TransmissionEbolaHemorrhagic,KhanEtAl_1999_ReemergenceEbolaHemorrhagic,BwakaEtAl_1999_EbolaHemorrhagicFever,NdambiEtAl_1999_EpidemiologicClinicalAspects} and estimates provided by \citet{LegrandEtAl_2007_UnderstandingDynamicsEbola}.
The estimates of \citet{LegrandEtAl_2007_UnderstandingDynamicsEbola} are maximum likelihood estimates fitted on morbidity data from the 1995 outbreak.

Let \(Y_t^{(M)} = \left(S_t / M, E_t / M, I_t / M, H_t / M, F_t / M\right)^{\T}\) denote the proportion of individuals in each compartment after \(t\) weeks.
The population process is also density dependent, with
\[
	f\!\left(\left(s, e, i, h, d\right), l\right) = \begin{cases}
		\beta_I s i + \beta_H sh + \beta_F sd,                        & \text{if } l = \left(-1, 1, 0, 0, 0\right), \\
		\alpha e,                                                     & \text{if } l = \left(0, -1, 1, 0, 0\right), \\
		\gamma_h \theta_1 i,                                          & \text{if } l = \left(0, 0, -1, 1, 0\right), \\
		\gamma_{dh}\delta_2 h,                                        & \text{if } l = \left(0, 0, 0, -1, 1\right), \\
		\gamma_f d,                                                   & \text{if } l = \left(0, 0, 0, 0, -1\right), \\
		\gamma_i\left(1 - \theta_1\right)\left(1 - \delta_1\right) i, & \text{if } l = \left(0,0,-1,0,0\right),     \\
		\delta_1\left(1 - \theta_1\right)\gamma_d i,                  & \text{if } l = \left(0,0,-1,0,1\right),     \\
		\gamma_{ih}\left(1 - \delta_2\right)h,                        & \text{if } l = \left(0,0,0,-1,0\right),     \\
		0,                                                            & \text{otherwise}.
	\end{cases}
\]
The deterministic differential equation is
\[
	\dod{Y_t^{(\infty)}}{t} = \begin{bmatrix}
		-\beta_I Y_t^{(\infty, 1)}Y_t^{(\infty, 3)} - \beta_H Y_t^{(\infty, 1)}Y_t^{(\infty, 4)} - \beta_F Y_t^{(\infty, 1)}Y_t^{(\infty,5)}                                                \\
		\beta_I Y_t^{(\infty, 1)}Y_t^{(\infty, 3)} + \beta_H Y_t^{(\infty, 1)}Y_t^{(\infty, 4)} + \beta_F Y_t^{(\infty, 1)}Y_t^{(\infty,5)} - \alpha Y_t^{(\infty, 2)}                      \\
		\alpha Y_t^{(\infty, 2)} - \left(\gamma_h\theta_1 + \gamma_i\left(1 - \theta_1\right)\left(1 - \delta_1\right) + \delta_1\left(1 - \theta_1\right)\gamma_d\right) Y_t^{(\infty, 3)} \\
		\gamma_h\theta_1 Y_t^{(\infty, 3)} - \left(\gamma_{dh}\delta_2 + \gamma_{ih}\left(1 - \delta_2\right) \right) Y_t^{(\infty, 4)}                                                     \\
		\gamma_{dh}\delta_2 Y_t^{(\infty, 4)} + \delta_1\left(1 - \theta_1\right)\gamma_d Y_t^{(\infty, 3)} - \gamma_f Y_t^{(\infty, 5)}
	\end{bmatrix}.
\]
The \(5\times 5\) diffusion matrix \(g\) satisfies

\begin{scriptaligned}
	\left[g\!\left(Y_t^{(\infty)}\right)g\!\left(Y_t^{(\infty)}\right)^{\T}\right]_{11} & =	\beta_I Y_t^{(\infty, 1)} Y_t^{(\infty, 3)} + \beta_H Y_t^{(\infty, 1)} Y_t^{(\infty, 4)} + \beta_F Y_t^{(\infty, 1)} Y_t^{(\infty, 5)} \\
	\left[g\!\left(Y_t^{(\infty)}\right)g\!\left(Y_t^{(\infty)}\right)^{\T}\right]_{12} & = \left[g\!\left(Y_t^{(\infty)}\right)g\!\left(Y_t^{(\infty)}\right)^{\T}\right]_{21} =  -\beta_I Y_t^{(\infty, 1)} Y_t^{(\infty, 3)} - \beta_H Y_t^{(\infty, 1)} Y_t^{(\infty, 4)} - \beta_F Y_t^{(\infty, 1)} Y_t^{(\infty, 5)} \\
	\left[g\!\left(Y_t^{(\infty)}\right)g\!\left(Y_t^{(\infty)}\right)^{\T}\right]_{22} & = \beta_I Y_t^{(\infty, 1)} Y_t^{(\infty, 3)} + \beta_H Y_t^{(\infty, 1)} Y_t^{(\infty, 4)} + \beta_F Y_t^{(\infty, 1)} Y_t^{(\infty, 5)} +  \alpha Y_t^{(\infty, 2)} \\
	\left[g\!\left(Y_t^{(\infty)}\right)g\!\left(Y_t^{(\infty)}\right)^{\T}\right]_{23} & = \left[g\!\left(Y_t^{(\infty)}\right)g\!\left(Y_t^{(\infty)}\right)^{\T}\right]_{32} = -\alpha Y_t^{(\infty, 2)} \\
	\left[g\!\left(Y_t^{(\infty)}\right)g\!\left(Y_t^{(\infty)}\right)^{\T}\right]_{33} & = \alpha Y_t^{(\infty, 2)} + \left(\gamma_H \theta_1  + \gamma_i(	1 - \theta_1)(1 - \delta_1) + \delta_1(1 - \theta_1)\gamma_d\right)Y_t^{(\infty, 1)} \\
	\left[g\!\left(Y_t^{(\infty)}\right)g\!\left(Y_t^{(\infty)}\right)^{\T}\right]_{34} & = \left[g\!\left(Y_t^{(\infty)}\right)g\!\left(Y_t^{(\infty)}\right)^{\T}\right]_{43} = - \gamma_H \theta_1 Y_t^{(\infty, 1)}\\
	\left[g\!\left(Y_t^{(\infty)}\right)g\!\left(Y_t^{(\infty)}\right)^{\T}\right]_{35} & = \left[g\!\left(Y_t^{(\infty)}\right)g\!\left(Y_t^{(\infty)}\right)^{\T}\right]_{53} = - \delta_1(1 - \theta_1)\gamma_d Y_t^{(\infty, 1)}\\
	\left[g\!\left(Y_t^{(\infty)}\right)g\!\left(Y_t^{(\infty)}\right)^{\T}\right]_{44} & = \gamma_H \theta_1 Y_t^{(\infty, 1)} + \left(\gamma_{dh}\delta_2  + \gamma_{ih}(1 - \delta_2)\right)Y_t^{(\infty, 4)} \\
	\left[g\!\left(Y_t^{(\infty)}\right)g\!\left(Y_t^{(\infty)}\right)^{\T}\right]_{45} & = \left[g\!\left(Y_t^{(\infty)}\right)g\!\left(Y_t^{(\infty)}\right)^{\T}\right]_{54} = -\gamma_{dh}\delta_2 Y_t^{(\infty, 4)} \\
	\left[g\!\left(Y_t^{(\infty)}\right)g\!\left(Y_t^{(\infty)}\right)^{\T}\right]_{55} & = \delta_1(1 - \theta_1)\gamma_d Y_t^{(\infty, 1)} + \gamma_{dh}\delta_2 Y_t^{(\infty, 4)} + \gamma_f Y_t^{(\infty, 5)}
\end{scriptaligned}
with all other entries zero.
There are many choices of \(g\) such that the product \(gg^{\T}\) has these entries, but to compute the Gaussian approximation with Mazzoni's method, we only need to evaluate \(g\!\left(Y_t^{(\infty)}\right)g\!\left(X_t^{(\infty)}\right)\) and therefore no not need to make such a choice.
