%!TEX root = ../thesis.tex

\chapter{Background}


\section{Notation}


The norm symbol \(\norm{\cdot}\) without any additional qualifiers denotes the standard Euclidean norm for a vector, and the spectral (operator) norm induced by the Euclidean norm, i.e. for an \(n \times n\) matrix \(A\)
\[
	\norm{A} = \sup\left\{\frac{\norm{Av}}{\norm{v}}\,\middle|\,v \in \R^n, \, \norm{v} \neq 0\right\}.
\]


For example, we say that a random variable \(A\) is equal to another random variable \(B\) almost surely (a.s.) if \(P\left(A = B\right) = 1\).




\section{Results from dynamical systems}
\begin{equation}
	\dod{w_t}{t} = u\left(w_t, t\right), \qquad w_0 = x \in \Omega,
	\label{eqn:det_ode}
\end{equation}
where \(u: \Omega \times[0,T] \to \R^n\) describes the velocity at a point in space and time.

In the mathematical treatment of Lagrangian dynamics, and in particular Lagrangian coherent structures \citep{BalasuriyaEtAl_2018_GeneralizedLagrangianCoherent}, trajectories solving \eqref{eqn:det_ode} are summarised by the flow map.
The flow map is an operator mapping

The flow map can be defined formally as follows.
\begin{definition}[Flow map]
	Suppose \(t_1, t_2 \in [0,T]\).
	The \textbf{flow map} \(F_{t_1}^{t_2}: \R^n \to \R^n\) from time \(t_1\) to \(t_2\) associated with \eqref{eqn:det_ode} is the unique solution to
	\begin{equation}
		\dpd{F_{t_1}^{\tau}(x)}{\tau} = u\left(F_{t_1}^\tau(x), \tau\right), \qquad F_{t_1}^{t_1}(x) = x,
		\label{eqn:flow_map_ode}
	\end{equation}
	solved up to time \(\tau = t_2\).
	Equivalently,
	\[
		F_{t_1}^{t_2}(x) = x + \int_{t_1}^{t_2}{u\left(F_{t_1}^\tau(x), \tau\right)\dif\tau}.
	\]
\end{definition}
The flow map satisfies the following properties, under ASSUMPTIONS? 
For any \(r, s, t \in [0,T]\) and points \(x,w \in \R^n\),
\begin{enumerate}
	\item \(F_{s}^{t}\) is invertible with inverse
		\[
			\left[F_{s}^{t}\right]^{-1}\left(w\right) = F_{t}^{s}\left(w\right).
		\]
	\item \(F_s^{t}\left(F_{r}^{s}(x)\right) = F_{r}^{t}\left(x\right)\) 
\end{enumerate}

The gradient of the flow map satisfies a useful property; the equation of variations.
\begin{theorem}
	Let \(F_{t_1}^{t}\) be the flow map corresponding to \eqref{eqn:det_ode}.
	Then, the spatial gradient \(\nabla F_{t_0}^t(x)\) satisfies the equation of variations
	\begin{equation}
		\dpd{\nabla F_{t_1}^{t}(x)}{t} = \nabla u\left(F_{t_1}^{t}(x), t\right) \nabla F_{t_1}^{t}(x).
		\label{eqn:eqn_of_variations}
	\end{equation}
\end{theorem}
\begin{proof}
	Taking the gradient on both sides of \eqref{eqn:flow_map_ode} and using the chain rule gives
	\[
		\nabla\left(\dpd{F_{t_1}^{t_2}(x)}{t}\right) = \nabla u\left(F_{t_1}^{t}(x), t\right) \nabla F_{t_1}^{t}(x).
	\]
	SMOOTHNESS
\end{proof}


% \subsection{Lagrangian coherent structures}
% \td{Briefly discuss LCSs. Do not need to go into too much detail here. Mention whatever is relevant}

An important inequality

\begin{theorem}[Gr\"onwall's inequality]\label{thm:gronwall}
	Let \(\alpha, \beta, u: [a,b] \to \R\) be functions such that \(\beta\) and \(u\) are continuous and that the negative part of \(\alpha\) is integrable on every closed and bounded subset of \([a,b]\).
	Then, if \(\beta\) is non-negative and for all \(t \in [a,b]\),
	\[
		u(t) \leq \alpha(t) + \int_a^t{\beta(\tau)u(\tau)\dif\tau}
	\]
	then
	\[
		u(t) \leq \alpha(t) + \int_a^t{\alpha(\tau)\beta(\tau)\exp\left(\int_\tau^{t}{\beta(s)\dif s}\right)\dif\tau}.
	\]
	Additionally, if \(\alpha\) is non-decreasing, then
	\[
		u(t) \leq \alpha(t) \exp\left(\int_a^t{\beta(\tau)\dif\tau}\right)
	\]
\end{theorem}



\section{Stochastic differential equations}

For an introduction to stochastic differential equations, see \citet{Oksendal_2003_StochasticDifferentialEquations} or \citet{KallianpurSundar_2014_StochasticAnalysisDiffusion}.


\begin{definition}[Canonical Wiener process]\label{def:wiener}

\end{definition}


The differential form of an \(n\)-dimensional It\^o stochastic differential equation is
\begin{equation}
	\dif y_t = u\left(y_t, t\right)\dif t + \sigma\left(y_t, t\right)\dif W_t,
	\label{eqn:gen_sde}
\end{equation}
where the solution \(y_t\) is a stochastic process taking values in \(\R^n\), \(u: \R^n \times \R \to \R^n\) is the drift and \(\sigma: \R^n \times \R \to \R^{n\times n}\) is the diffusivity.
The driving process \(W_t\) is the canonical, \(n\)-dimensional Wiener process as defined in \Cref{def:wiener}.
For a (possibly random) initial condition \(y_0\), the solution \(y_t\) to \eqref{eqn:gen_sde} satisfies
\begin{equation}
	y_t = y_0 + \int_0^t{u\left(y_\tau, \tau\right)\dif\tau} + \int_0^t{\sigma\left(y_\tau, \tau\right)\dif W_\tau}.
	\label{eqn:gen_sde_int}
\end{equation}
In the most general case, the drift \(u\) and diffusivity \(\sigma\) are permitted to themselves be random functions \cite{KallianpurSundar_2014_StochasticAnalysisDiffusion}, but in this thesis we assumed that both are deterministic.

As with deterministic differential equations, we are interested in knowing precisely when solutions to \eqref{eqn:gen_sde} exist, and whether these solutions are unique.

A solution to the stochastic differential equation can be defined rigorously \cite{KallianpurSundar_2014_StochasticAnalysisDiffusion}.
\begin{definition}
	A stochastic process \(\set{y_t}_{t \in [0,T]}\) taking values in \(R^n\) is said to be a \emph{strong solution} of \eqref{eqn:gen_sde} with initial condition \(y_0 = \xi\) if the following holds:
	\begin{enumerate}
		\item For each \(t\),
		\item
		      \[
			      \int_0^T{\left(\norm{u\left(y_t, t\right)} + \norm{\sigma\left(y_t, t\right)}^2\right)\dif t} < \infty \alsu
		      \]

		\item For each \(t \in [0,T]\),
		      \[
			      y_t = \xi + \int_0^t{u\left(y_\tau, \tau\right)\dif\tau} + \int_0^t{\sigma\left(y_\tau, \tau\right)\dif W_\tau} \quad \text{a.s.}
		      \]
	\end{enumerate}
\end{definition}

Solutions are known as \emph{strongly unique} if, for any two strong solutions \(x_t\) and \(y_t\) to \eqref{eqn:gen_sde} defined on the same measure space,
\[
	P\left(x_t = y_t, \quad \forall t \in \left[0,T\right]\right) = 1
\]
That is, \(x_t\) and \(y_t\) are equal almost surely.

We can now state the requirements for \eqref{eqn:gen_sde} to have unique solutions.
\begin{theorem}[\citet{KallianpurSundar_2014_StochasticAnalysisDiffusion}]
	Let \(u_i\) be the \(i\)th component of \(u\) and \(\sigma_{ij}\) the \((i,j)\)th component of \(\sigma\), for \(i,j = 1,\hdots,n\).
	Suppose that for each \(i,j = 1,\hdots,n\),
	\begin{romanate}
		\item \(u_i\) is Borel-measurable, and
		\item \(\sigma_{ij}\) is Borel-measurable.
	\end{romanate}
	Suppose further that for all \(t \in [0,T]\) and \(x,y \in \R^n\), there is a positive constant \(K\) independent of \(t\) and \(x, y\) such that
	\begin{romanate}
		\setcounter{enumi}{2}
		\item \(\norm{u(x,t)}^2 + \norm{\sigma(x,t)}^2 \leq K\left(1 + \norm{x}^2\right)\) a.s.
		\item \(\norm{u\left(x,t\right) - u\left(y,t\right)}^2 + \norm{\sigma\left(x,t\right) - \sigma\left(y,t\right)}^2 \leq K\norm{x - y}\) a.s.
	\end{romanate}
	Then, \eqref{eqn:gen_sde} has a strongly unique strong solution.
\end{theorem}
\begin{proof}
	See Theorem 6.2.1 of \cite{KallianpurSundar_2014_StochasticAnalysisDiffusion}, for instance.
\end{proof}




\subsection{Analytical tools for It\^o calculus}
There are several tools available for the analytic treatment of It\^o integrals and solutions to stochastic differential equations, which we make use of throughout.
The first is It\^o's Lemma (or the It\^o Formula), which is a change-of-variables formula in stochastic calculus and can be thought of as a generalisation of the chain rule from deterministic calculus.
We state and use the multidimensional form of the Lemma for solutions to It\^o stochastic differential equations, although more general forms exist (e.g. see Theorem 5.4.1 of \cite{KallianpurSundar_2014_StochasticAnalysisDiffusion}).
\begin{theorem}[It\^o's Lemma]
	Let \(X_t\) be the strong solution to the stochastic differential equation
	\[
		\dif X_t = a\left(X_t, t\right)\dif t + b\left(X_t, t\right)\dif W_t,
	\]
	where \(a: \R^n \times [0,\infty) \to \R^n\), \(b: \R^n \times [0,\infty) \to \R^{n\times p}\) and \(W_t\) is the canonical \(p\)-dimensional Wiener process.
	If \(f: \R^n \times [0, \infty) \to \R^m\) is twice continuously-differentiable, then the stochastic process \(Y_t \coloneqq f\left(X_t, t\right)\) is a strong solution to the stochastic differential equation
	\begin{align*}
		\dif Y_t & = \left(\dpd{f}{t}\left(X_t, t\right) + \nabla f\left(X_t, t\right) a\left(X_t, t\right) + \frac12\mathrm{tr}\left[b\left(X_t, t\right)^{\T} \nabla\nabla f\left(X_t, t\right) b\left(X_t, t\right)\right] \right)\dif t \\
		         & \qquad\qquad\qquad\qquad + \nabla f\left(X_t, t\right)b\left(X_t, t\right) \dif W_t.
	\end{align*}
\end{theorem}
\begin{proof}

\end{proof}



\begin{theorem}[Burkholder-Davis-Gundy Inequality]
	Let \(M_t\) be an It\^o-integrable stochastic process taking values in \(\R^n\).
	Then, for any \(p > 0\) there exists constants \(c_p, C_p > 0\) independent of the stochastic process \(M_t\) such that
	\[
		c_p\avg{\left(\int_0^t{\norm{M_\tau}^2\dif \tau}\right)^p} \leq \avg{\sup_{\tau \in \left[0, t\right]}\norm{\int_0^\tau{M_s\dif W_s}}^{2p}} \leq C_p\avg{\left(\int_0^t{\norm{M_\tau}^2\dif \tau}\right)^p}.
	\]
\end{theorem}
\begin{proof}

\end{proof}


\section{Aspects of stochastic parameterisation}
When using deterministic systems to model real-world phenomena, there are many ways in which uncertainty can arise.
For instance,

Stochastic parameterisation is a
These unresolved subgrid effects are accounted for by introducing stochastic noise into the otherwise deterministic model.

\citet{BernerEtAl_2017_StochasticParameterizationNew}

\citet{LeutbecherEtAl_2017_StochasticRepresentationsModel}

\subsection{Additive versus multiplicative noise}

When \(\sigma = \sigma(t)\) depends only on \(t\), then noise is considered \emph{additive}.
If there is spatial dependence in \(\sigma\), i.e. \(\sigma = \sigma(x,t)\), then the noise considered \emph{multiplicative}.



For instance, \citet{SuraEtAl_2005_MultiplicativeNoiseNonGaussianity} shows that the non-Gaussian statistics observed in atmospheric regimes can arise from linear models with multiplicative noise.




\section{Stochastic sensitivity}
In most practical situations, the Eulerian velocity data driving ocean and atmospheric models relies upon measurements of estimates obtained on a low resolution spatial discretisation.


There are limited tools within the LCS context that explcitly characterise the impact of these uncertainties
As such, there is recent interest in addressing this deficiency \citep{BalasuriyaGottwald_2018_EstimatingStableUnstablea, Balasuriya_2020_StochasticApproachesLagrangian}\lb{Probably need some non-Sanjeeva citations here}.
\citet{Balasuriya_2020_StochasticSensitivityComputable} introduces stochastic sensitivity as a new tool for directly quantifying the impact of Eulerian uncertainty on Lagrangian trajectories.
The evolution of Lagrangian trajectories is modelled as solution to a It\^o stochastic ordinary differential equation.

\td{Deterministic}

The SDE model is
\begin{equation}
	\sde{y_t}{u\left(y_t, t\right)}{\epsilon\sigma\left(y_t, t\right)},
	\label{eqn:ss_sde},
\end{equation}
where \(0 < \epsilon \ll 1\) is a parameter quantifying the scale of the noise, \(\sigma:	\R^2\times[0,T] \to \R^{2\times 2}\) is the \(2\times 2\) diffusion matrix, and \(W_t\) is the canonical two-dimensional Wiener process.
Since \(\sigma\) can vary by both space and time, the noise is multiplicative.

To quantify uncertainty in a way that is independent of the noise scale \(\epsilon\), \cite{Balasuriya_2020_StochasticSensitivityComputable} defined the random variable \(z_\epsilon\left(x,t\right)\) on \(\R^2 \times [0,T]\) as
\[
	z_\epsilon\left(x,t\right) \coloneqq \frac{y_t - F_0^t(x)}{\epsilon}.
\]
The main aim is to compute statistics of \(z_\epsilon\) at the final time \(T\), so that of \(z_\epsilon\left(x,T\right)\).
\citet{Balasuriya_2020_StochasticSensitivityComputable} then considers the signed projection of \(z_\epsilon\left(x,T\right)\) onto a ray emanating from the deterministic position \(F_0^T(x)\) in a given direction, defining
\[
	P_\epsilon\left(x,\theta\right) \coloneqq \hat{n}^{\T} z_\epsilon(x,T),
\]
where \(\theta \in \left[-\pi/2, \pi/2\right)\) and
\[
	\hat{n}(\theta) = \begin{bmatrix}
		\cos{\theta} \\
		\sin{\theta}
	\end{bmatrix}.
\]
The statistics of \(z_\epsilon\left(x,T\right)\) and \(P_\epsilon(x,\theta)\) are considered in the limit as \(\epsilon\downarrow 0\), which \td{something}.

The first result established by \cite{Balasuriya_2020_StochasticSensitivityComputable} is that the expected location is deterministic, in the following sense.
\begin{theorem}[\cite{Balasuriya_2020_StochasticSensitivityComputable}]
	For all \(x \in \R^2\),
	\[
		\lim_{\epsilon\downarrow 0}\avg{z_\epsilon(x,T)} = 0.
	\]
\end{theorem}

The variance of \(P_\epsilon\left(x,\theta\right)\) is used to assign a computable scalar measure of uncertainty to the trajectory.

\begin{definition}[\cite{Balasuriya_2020_StochasticSensitivityComputable}]
	\begin{alpharate}
		\item The \textbf{anisotropic uncertainty} is a scalar field \(A: \R^2\times\left[-\pi/2, \pi/2\right) \to [0,\infty)\) defined by
		\[
			A(x,\theta) \coloneqq \sqrt{\lim_{\epsilon\downarrow 0}\var{P_\epsilon(x,\theta)}}.
		\]

		\item The \textbf{stochastic sensitivity} is a scalar field \(S: \R^2 \to [0,\infty)\) defined by
		\[
			S^2(x) \coloneqq \lim_{\epsilon\downarrow 0}\sup_{\theta}{\var{P_\epsilon(x,\theta)}}.
		\]
	\end{alpharate}
\end{definition}

By employing techniques from both deterministic and stochastic calculus (i.e. Gr\"onwall's inequality, the Burkholder-Davis-Gundy inequality, It\^o's Lemma), Balasuriya further established expressions for both the anisotropic uncertainty and the stochastic sensitivity that are computable given only the flow map and velocity data.

\begin{theorem}[\cite{Balasuriya_2020_StochasticSensitivityComputable}]
	For \(x \in \R^2\), set \(w \coloneqq F_0^t(x)\).
	Then, for any \(\theta \in \left[-\pi/2, \pi/2\right)\),
	\[
		A(x,\theta) = \left(\int_0^T{\norm{\Lambda\left(x, t, T\right)J\hat{n}(\theta)}\dif t}\right)^{1/2},
	\]
	where
	\[
		\Lambda\left(x,t,T\right) \coloneqq e^{\int_t^T{\left[\nabla \cdot u\right]\left(F_0^\xi(x), \xi\right)\dif\xi}}\sigma\left(F_0^t(x), t\right)^{\T}J \nabla_w F_T^t\left(w\right),
	\]
	with the gradients \(\nabla_w\) of the flow map taken with respect to the mapped position \(w\), and
	\[
		J \coloneqq \begin{bmatrix}
			0 & -1 \\
			1 & 0
		\end{bmatrix}
	\]
	Additionally, stochastic sensitivity is computed as
	\[
		S^2(x) = P(x) + N(x),
	\]
	with
	\begin{align*}
		L(x) & \coloneqq \frac12\sum_{i=1}^2\int_0^T\left[\Lambda_{i2}\left(x,t, T\right)^2 - \Lambda_{i1}\left(x,t,T\right)^2\right]\dif t \\
		M(x) & \coloneqq \sum_{i=1}^2\int_0^T{\Lambda_{i1}\left(x,t,T\right)\Lambda_{i2}\left(x,t,T\right)\dif t}                           \\
		N(x) & \coloneqq \sqrt{L^2(x) + M^2(x)}                                                                                             \\
		P(x) & \coloneqq \abs{\frac12\sum_{i=1}^2\sum_{j=1}^2{\int_0^T{\Lambda_{ij}\left(x,t,T\right)^2\dif t}}},
	\end{align*}
	where \(\Lambda_{ij}\) is the \((i,j)\)-element of \(\Lambda\).
\end{theorem}


\subsection{Current applications \& shortcomings}
Since stochastic sensitivity is only a recent development, it has only been applied in a limited number of places so far.
Here, we briefly review the literature in which the original formulation stochastic sensitivity by \cite{Balasuriya_2020_StochasticSensitivityComputable} has been applied.

\begin{itemize}
	\item \citet{Balasuriya_2020_UncertaintyFinitetimeLyapunov} uses stochastic sensitivity to compute an error bound for the finite-time Lyapunov computation.


	\item \citet{FangEtAl_2020_DisentanglingResolutionPrecision}


	\item \citet{BadzaEtAl_2023_HowSensitiveAre} investigate the impact of velocity uncertainty on Lagrangian coherent structures (e.g. see the reviews by \citet{BalasuriyaEtAl_2018_GeneralizedLagrangianCoherent} and \citet{HadjighasemEtAl_2017_CriticalComparisonLagrangian}) extracted as robust sets with stochastic sensitivity.

\end{itemize}

Although

\begin{enumerate}
	\item The tools are restricted to two-dimensional models, and the constructions using projections have no obvious extension to \(n\)-dimensions.
	      Extending stochastic sensitivity to \(n\)-dimensions will enable application to a much broader class of models beyond the fluid flow context, including high-dimensional climate and ??? models.

	\item \citet{Balasuriya_2020_StochasticSensitivityComputable} only computes the expectation and variance of the projections \(P_\epsilon(x,\theta)\), which does not give us the distribution under the limit as \(\epsilon\) approaches 0.

	\item The computational formula for the anisotropic uncertainty and stochastic sensitivity require knowledge of the divergence \(\nabla\cdot u\) of the velocity field.
\end{enumerate}



