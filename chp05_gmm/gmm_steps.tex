\documentclass[crop,tikz]{standalone}

\usetikzlibrary{calc}

\begin{document}
\begin{tikzpicture}
	% STEP 1: PROPAGATE
	% Trajectory
	\draw (0,0) .. controls (2,2) and (4,-3) .. (6,0);

	% Initial position
	\fill[black] (0,0) circle[radius=0.5pt] node[anchor = east] {\(x\)};

	% Mapped position
	\fill[black] (6,0) circle[radius=0.5pt] node[anchor = west] {\(F_0^t\!\left(x\right)\)};

	% Covariance ellipse
	\draw[dashed, rotate around = {30: (6,0)}, red] (6,0) ellipse (30pt and 70pt);
	% \draw[dashed, rotate around = {30: (8,0)}, red] (8,0) ellipse (20pt and 50pt);
	% \draw[dashed, rotate around = {30: (8,0)}, red] (8,0) ellipse (10pt and 30pt);
	\node[red] at ($(6,0)+(75:30pt and 70pt)$) {\(\Sigma_0^t\!\left(x\right)\)};

	% STEP 2: Splitting
	% Trajectory
	\draw (0,-5) .. controls (2,-3) and (4,-8) .. (6,-5);

	% Initial position
	\fill[black] (0,-5) circle[radius=0.5pt] node[anchor = east] {\(x\)};

	% Mapped position
	\fill[blue] (6,-5) circle[radius=1pt];

	% Covariance ellipse
	\draw[dashed, rotate around = {30: (6,-5)}, red] (6,-5) ellipse (30pt and 70pt);

	% Additional sigma points
	\fill[blue, rotate around = {30: (6,-5)}] ($(6, -5)+(0:30pt and 70pt)$) circle[radius=1pt];
	\fill[blue, rotate around = {30: (6,-5)}] ($(6, -5)+(90:30pt and 70pt)$) circle[radius=1pt];
	\fill[blue, rotate around = {30: (6,-5)}] ($(6, -5)+(180:30pt and 70pt)$) circle[radius=1pt];
	\fill[blue, rotate around = {30: (6,-5)}] ($(6, -5)+(270:30pt and 70pt)$) circle[radius=1pt];

	% Covariances for each sigma point
	\draw[dashed, rotate around = {30: (6,-5)}, blue] (6, -5) ellipse (12pt and 28pt);
	\draw[dashed, rotate around = {30: (6,-5)}, blue] ($(6, -5)+(0:30pt and 70pt)$) ellipse (12pt and 28pt);
	\draw[dashed, rotate around = {30: (6,-5)}, blue] ($(6, -5)+(90:30pt and 70pt)$) ellipse (12pt and 28pt);
	\draw[dashed, rotate around = {30: (6,-5)}, blue] ($(6, -5)+(180:30pt and 70pt)$) ellipse (12pt and 28pt);
	\draw[dashed, rotate around = {30: (6,-5)}, blue] ($(6, -5)+(270:30pt and 70pt)$) ellipse (12pt and 28pt);


\end{tikzpicture}
\end{document}
