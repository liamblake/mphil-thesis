%!TEX root = ../thesis.tex

\chapter{Background}


\section{Notation}


For a



\section{Lagrangian Dynamics}


We are typically interested in a spatial domain \(\Omega \subseteq \R^n\) and over some finite time interval \([0,T]\) for finite \(T\).
Lagrangian trajectories are solutions of the first-order differential equation
\begin{equation}
	\dod{x_t}{t} = u\left(x_t, t\right), \qquad x_0 = x \in \Omega,
	\label{eqn:det_ode}
\end{equation}
where \(u: \Omega\times[0,T] \to \R^n\) describes the velocity at a point in space and time.





\subsection{Flow map}
In the mathematical treatment of Lagrangian dynamics, and in particular Lagrangian coherent structures \citep{BalasuriyaEtAl_2018_GeneralizedLagrangianCoherent}, trajectories solving \eqref{eqn:det_ode} are summarised by the flow map.
The flow map is an operator mapping

The flow map can be defined more formally as follows.
\begin{defn}[Flow map]
	Suppose \(t_1, t_2 \in [0,T]\).
	The \textbf{flow map} \(F_{t_1}^{t_2}: \R^n \to \R^n\) from time \(t_1\) to \(t_2\) associated with \eqref{eqn:det_ode} is the solution to
	\[
		\dpd{F_{t_1}^{\tau}(x)}{\tau} = u\left(F_{t_1}^\tau(x), \tau\right), \qquad F_{t_1}^{t_1}(x) = x,
	\]
	solved up to time \(\tau = t_2\).
	Equivalently,
	\[
		F_{t_1}^{t_2}(x) = x + \int_{t_1}^{t_2}{u\left(F_{t_1}^\tau(x), \tau\right)\dif\tau}.
	\]
\end{defn}
The flow map satisfies the following properties


The gradient of the flow map satisfies a useful property; the equation of variations.
\begin{theorem}
	Let \(F_{t_0}^{t}\) be the flow map corresponding to \eqref{eqn:det_ode}.
	Then, the spatial gradient \(\nabla F_{t_0}^t(x)\) satisfies the equation of variations
	\begin{equation}
		\dpd{\nabla F_{t_0}^{t}(x)}{t} = \nabla u\left(F_{t_0}^t(x), t\right) \nabla F_{t_0}^t(x).
		\label{eqn:eqn_of_variations}
	\end{equation}
\end{theorem}
\begin{proof}

\end{proof}


\subsection{Lagrangian coherent structures}
\td{Briefly discuss LCSs. Do not need to go into too much detail here. Mention whatever is relevant}


\section{Stochastic Differential Equations}


\begin{equation}
	\dif y_t = u\left(y_t, t\right)\dif t + \sigma\left(y_t, t\right)\dif W_t.
	\label{eqn:gen_sde}
\end{equation}

A solution to the stochastic differential equation can be defined rigorously \cite{KallianpurSundar_2014_StochasticAnalysisDiffusion}.
\begin{defn}
	A stochastic process \(\set{y_t}_{t \in [0,T]}\) taking values in \(R^n\) is said to be a \textbf{strong solution} of \eqref{eqn:gen_sde} with initial condition \(y_0 = \xi\) if the following holds:
	\begin{enumerate}
		\item For each \(t\),
		\item
		      \[
			      \int_0^T{\left(\norm{u\left(y_t, t\right)} + \norm{\sigma\left(y_t, t\right)}^2\right)\dif t} < \infty \alsu
		      \]

		\item For each \(t \in [0,T]\),
	\end{enumerate}
\end{defn}


\section{Stochastic Sensitivity}
In most practical situations, the Eulerian velocity data driving ocean and atmospheric models relies upon measurements of estimates obtained on a low resolution spatial discretisation.


There are limited tools within the LCS context that explcitly characterise the impact of these uncertainties
As such, there is recent interest in addressing this deficiency \citep{BalasuriyaGottwald_2018_EstimatingStableUnstablea, Balasuriya_2020_StochasticApproachesLagrangian}\lb{Probably need some non-Sanjeeva citations here}.
\cite{Balasuriya_2020_StochasticSensitivityComputable} introduces stochastic sensitivity as a new tool for directly quantifying the impact of Eulerian uncertainty on Lagrangian trajectories.
The evolution of Lagrangian trajectories is modelled as solution to a It\^o stochastic ordinary differential equation.

\td{Deterministic}

The SDE model is
\begin{equation}
	\sde{y_t}{u\left(y_t, t\right)}{\epsilon\sigma\left(y_t, t\right)},
	\label{eqn:ss_sde},
\end{equation}
where \(0 < \epsilon \ll 1\) is a parameter quantifying the scale of the noise, \(\sigma:	\R^2\times[0,T] \to \R^{2\times 2}\) is the \(2\times 2\) diffusion matrix, and \(W_t\) is the canonical two-dimensional Wiener process.
Since \(\sigma\) can vary by both space and time, the noise is multiplicative.

To quantify uncertainty in a way that is independent of the noise scale \(\epsilon\), \cite{Balasuriya_2020_StochasticSensitivityComputable} defined the random variable \(z_\epsilon\left(x,t\right)\) on \(\R^2 \times [0,T]\) as
\[
	z_\epsilon\left(x,t\right) \coloneqq \frac{y_t - F_0^t(x)}{\epsilon}.
\]
The main aim is to compute statistics of \(z_\epsilon\) at the final time \(T\), so that of \(z_\epsilon\left(x,T\right)\).
\cite{Balasuriya_2020_StochasticSensitivityComputable} then considers the signed projection of \(z_\epsilon\left(x,T\right)\) onto a ray emanating from the deterministic position \(F_0^T(x)\) in a given direction, defining
\[
	P_\epsilon\left(x,\theta\right) \coloneqq \hat{n}^{\T} z_\epsilon(x,T),
\]
where \(\theta \in \left[-\pi/2, \pi/2\right)\) and
\[
	\hat{n}(\theta) = \begin{bmatrix}
		\cos{\theta} \\
		\sin{\theta}
	\end{bmatrix}.
\]
The statistics of \(z_\epsilon\left(x,T\right)\) and \(P_\epsilon(x,\theta)\) are considered in the limit as \(\epsilon\downarrow 0\), which \td{something}.

The first result established by \cite{Balasuriya_2020_StochasticSensitivityComputable} is that the expected location is deterministic, in the following sense.
\begin{theorem}[\cite{Balasuriya_2020_StochasticSensitivityComputable}]
	For all \(x \in \R^2\),
	\[
		\lim_{\epsilon\downarrow 0}\avg{z_\epsilon(x,T)} = 0.
	\]
\end{theorem}

The variance of \(P_\epsilon\left(x,\theta\right)\) is used to assign a computable scalar measure of uncertainty to the trajectory.

\begin{defn}[\cite{Balasuriya_2020_StochasticSensitivityComputable}]
	\begin{alpharate}
		\item The \textbf{anisotropic uncertainty} is a scalar field \(A: \R^2\times\left[-\pi/2, \pi/2\right) \to [0,\infty)\) defined by
		\[
			A(x,\theta) \coloneqq \sqrt{\lim_{\epsilon\downarrow 0}\var{P_\epsilon(x,\theta)}}.
		\]

		\item The \textbf{stochastic sensitivity} is a scalar field \(S: \R^2 \to [0,\infty)\) defined by
		\[
			S^2(x) \coloneqq \lim_{\epsilon\downarrow 0}\sup_{\theta}{\var{P_\epsilon(x,\theta)}}.
		\]
	\end{alpharate}


	\begin{theorem}[\cite{Balasuriya_2020_StochasticSensitivityComputable}]
		For \(x \in \R^2\) and \(\theta \in \left[-\pi/2, \pi/2\right)\),
		\[
			A(x,\theta) = \left(\int_0^T{\norm{\Lambda\left(F_0^t(x), t\right)J\hat{n}(\theta)}\dif t}\right)^{1/2}
		\]
	\end{theorem}



\end{defn}


