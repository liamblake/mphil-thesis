\chapter{Characterising SDE linearisations: the theory}\label{ch:linear_theory}
In general, stochastic differential equations cannot be solved analytically and instead require computationally expensive numerical simulation.
An alternative approach is to approximate the SDE by a simplified one, which can be solved analytically.
A linearisation procedure is one such approach when the noise is small, in which the coefficients of the SDE are replaced with first-order Taylor expansions.
As we highlighted in the introduction (\Cref{ch:intro}), this linearisation scheme provides a practical approach to using SDEs and is accordingly used across a range of literature and applications \citep{Jazwinski_2014_StochasticProcessesFiltering,SarkkaSolin_2019_AppliedStochasticDifferential,KaszasHaller_2020_UniversalUpperEstimate,ArchambeauEtAl_2007_GaussianProcessApproximations,Sanz-AlonsoStuart_2017_GaussianApproximationsSmall,LawEtAl_2015_DataAssimilationMathematical,ReichCotter_2015_ProbabilisticForecastingBayesian,BudhirajaEtAl_2019_AssimilatingDataModels,LeGlandWang_2002_AsymptoticNormalityPartially}.
% Much is already known about the theory of these linearisations: classical results in the context of small-noise series expansions \citep{Blagoveshchenskii_1962_DiffusionProcessesDepending} and large deviations theory \citep{FreidlinWentzell_1998_RandomPerturbationsDynamical} show that the strong error between SDE solution with a fixed initial condition and that of an appropriate linearisation is bounded.
% \citet{Sanz-AlonsoStuart_2017_GaussianApproximationsSmall} establish a strong result, bounding the Kullback-Leibler divergence between the solutions of autonomous SDEs with additive stationary noise and a linearised equivalent. Their result considers both an uncertain initial condition, and the evolving error due to the discrepancy between the models.

In this chapter, we build upon previous small-noise studies of SDE linearisations \citep{Blagoveshchenskii_1962_DiffusionProcessesDepending,FreidlinWentzell_1998_RandomPerturbationsDynamical,Sanz-AlonsoStuart_2017_GaussianApproximationsSmall} to provide an explicit bound for the error between a general class of stochastic differential equations and linearisations thereof about trajectories of a corresponding deterministic system.
Our framework accounts for non-autonomous coefficients, multiplicative noise, and uncertain initial conditions in a stochastic differential equation of arbitrary dimension---a more general situation than that considered by any of \citet{Blagoveshchenskii_1962_DiffusionProcessesDepending}, \citet{FreidlinWentzell_1998_RandomPerturbationsDynamical}, or \citet{Sanz-AlonsoStuart_2017_GaussianApproximationsSmall}.
In \Cref{sec:theory}, we state the bound, written in terms of scales of the initial and the ongoing uncertainty, and provide an explicit characterisation of the solution to the linearised SDE including computations for the first two moments.
We directly compare our newly derived bound in \Cref{sec:comparison} to that on the KL-divergence by \citet{Sanz-AlonsoStuart_2017_GaussianApproximationsSmall} and postulate that our bound is tighter.

The second contribution of this chapter is to extend the original formulation of stochastic sensitivity by \citet{Balasuriya_2020_StochasticSensitivityComputable}, in \Cref{sec:theory_s2}.
We provide a new definition of stochastic sensitivity for \(n\)-dimensions and establish that the value can be computed from the solution of a linearised SDE.

Much of the content in this chapter and the following (\Cref{ch:linear_numerics}) has been submitted as a research article to \textit{Communications in Mathematical Sciences} \citep{BlakeEtAl_2023_ConvergenceStochasticDifferential}, and is currently under review.
The preprint is available on arXiv at \href{https://arxiv.org/abs/2309.16334}{arXiv:2309.16334}.
\Cref{sec:ftle_s2_connection}, which discusses the connections between stochastic sensitivity and the finite-time Lyapunov exponent, does not appear in the submitted article and is instead a new contribution in this thesis.


\section{Convergence of a SDE to a linearisation}\label{sec:theory}
Suppose we are interested in the evolution of a \(\R^n\)-valued state variable \(y_t\) over a finite time interval \([0,T]\).
Our model, accounting for uncertainties arising from a range of sources, for the evolution of this variable is the It\^o stochastic differential equation
\begin{equation}
	\dif y_t^{(\epsilon)} = u\!\left(y_t^{(\epsilon)}, t\right)\dif t + \epsilon \, \sigma\!\left(y_t^{(\epsilon)}, t\right)\dif W_t, \quad y_0^{(\epsilon)} = x
	\label{eqn:sde_y}
\end{equation}
where \(u\colon \R^n \times [0,T] \to \R^n\) is the governing reference vector field.
The canonical \(m\)-dimensional Wiener process \(W_t\)  is a continuous white-noise stochastic process with independent Gaussian increments.
The scale of the ongoing noise is assumed to be small and is parameterised as \(0 < \epsilon \ll 1\).
The noise in \cref{eqn:sde_y} is multiplicative, in that the diffusion matrix \(\sigma\colon \R^n \times [0,T] \to \R^{n\times m}\) can vary with state \( x \), as well as with time \( t \).
We assume that \(\sigma\) is specified \textit{a priori}, or if no such information is known, then \(\sigma \equiv I\), the \(n \times m\) identity matrix, is a default modelling choice.
We consider \cref{eqn:sde_y} subject to the \emph{general} uncertain initial condition \(y_0^{(\epsilon)} = x\), where \(x\) is an \(n\)-dimensional random vector with some given distribution. The two sources of randomness, $ x $ and $ W_t $, are assumed independent from each other.

In the absence of any uncertainty (i.e.\ \(\epsilon = 0\) and the initial condition is a known deterministic quantity), \cref{eqn:sde_y} reduces to the ordinary differential equation
\begin{equation}
	\dod{y_t^{(0)}}{t} = u\!\left(y_t^{(0)}, t\right), \quad y_0^{(0)} = x_0.
	\label{eqn:ode_det}
\end{equation}
where the initial condition \(x_0 \in \R^n\) is fixed.
The formal convergence of the stochastic solution \(y_t^{(\epsilon)}\) (under certain conditions on the initial condition) to the deterministic \(y_{t}^{(0)}\) in the limit as \(\epsilon \to 0\) is well-established using the large deviations principle \citep[e.g]{FreidlinWentzell_1998_RandomPerturbationsDynamical}.
We refer to \cref{eqn:ode_det} as the \emph{reference} deterministic model associated with \cref{eqn:sde_y}.
Solutions to the reference deterministic model are more readily available, e.g.\ in terms of computational efficiency when solving numerically, than those of the stochastic model, but do not account for inevitable uncertainty.

Let the flow map \(F_{0}^{t}\colon \R^n \to \R^n\) be the function which evolves an initial condition from time \(0\) to time \(t\) according to the flow of \cref{eqn:ode_det}, i.e.\ \(F_0^t\!\left(x_0\right) = y_t^{(0)}\).

We assume certain smoothness and boundedness conditions on the various terms outlined, which are stated explicitly in \Cref{hyp:smooth}.
Throughout this article, we use the norm symbol \(\norm{\cdot}\) to denote (i) for a vector, the standard Euclidean vector norm, (ii) for a matrix, the spectral norm induced by the Euclidean norm, and (iii) for a 3rd-order tensor, the spectral norm induced by the matrix norm.
The gradient symbol \(\nabla\) generically refers to derivatives with respect to the state variable.

\renewcommand\thehypo{H}
\begin{hypo}\label{hyp:smooth}
	Let the deterministic functions \(u \colon \R^n\times [0,T] \to \R^n\) and \(\sigma \colon \R^n \times [0,T] \to \R^{n\times m}\), and the random initial condition \(x\) be such that:
	\begin{enumerate}[label=(H.\arabic{*}), ref=H.\arabic{*}]
		\item\label{hyp:fm_exists} For all \(t \in [0,T]\) flow map \(F_0^t \colon \R^n \to \R^n\) is well-defined, and continuously differentiable (with respect to the initial condition) with invertible derivative.

		\item\label{hyp:coef_cont} For each \(t \in [0,T]\), the function \(u(\cdot, t): \R^n \to \R^n\) given by \(u(x,t)\) is twice continuously differentiable on \(\R^n\), and each component of the function \(\sigma(\cdot, t): \R^n \to \R^{n\times m}\) given by \(\sigma(x,t)\) is differentiable on \(\R^n\).

		\item\label{hyp:u_bounds} There exists a constant \(K_{\nabla u} \geq 0\) such that for any \(t \in [0,T]\) and \(x \in \R^n\),
		\begin{equation*}
			\norm{\nabla u(x,t)} \leq K_{\nabla u}.
		\end{equation*}
		Equivalently, for all \(t \in [0,T]\), the function \(u\!\left(\cdot, t\right)\) is Lipschitz continuous with Lipschitz constant \(K_{\nabla u}\).

		\item\label{hyp:coef_meas} For each \(x \in \R^n\), the function \(u(x,\cdot) \colon [0,T] \to \R^n\) and each component of the function \(\sigma(x,\cdot) \colon [0,T] \to \R^{n\times m}\) are Borel-measurable on \([0,T]\).

		\item\label{hyp:linear_growth} There exists a constant \(K_L\) such that for any \(t \in [0,T]\) and \(x \in \R^n\),
		\[
			\norm{u\left(x,t\right)} + \norm{\sigma\left(x,t\right)} \leq K_L\left(1 + \norm{x}\right).
		\]

		\item\label{hyp:sigma_deriv_bound} There exists a constant \(K_{\nabla\sigma} \geq 0\) such that for any \(t \in [0,T]\) and \(x \in \R^n\),
		\begin{equation*}
			\norm{\nabla\sigma(x,t)} \leq K_{\nabla\sigma},
		\end{equation*}
		and we take \(K_{\nabla\sigma} = 0\) if there is no spatial dependence in \(\sigma\).
		Equivalently, for all \(t \in [0,T]\), the function \(\sigma\!\left(x, \cdot\right)\) is Lipschitz continuous with Lipschitz constant \(K_{\nabla\sigma}\).

		\item\label{hyp:init_indep} The initial condition \(x\) is defined on the same probability space as \(W_t\), and is independent of \(W_t\) for all \(t \in [0,T]\).

		\item\label{hyp:nnu_bounds} There exists a constant \(K_{\nabla\nabla u} \geq 0\) such that for any \(t \in [0,T]\) and \(x \in \R^n\),
		\[
			\norm{\nabla \nabla u(x,t)} \leq K_{\nabla\nabla u},
		\]
		and we take \(K_{\nabla\nabla} = 0\) if the second spatial derivatives of \(u\) are all zero.

		\item\label{hyp:sigma_bounds} There exists a constant \(K_\sigma \geq 0\) such that for any \(t \in [0,T]\) and \(x \in \R^n\),
		\begin{equation*}
			\norm{\sigma(x,t)} \leq K_{\sigma}.
		\end{equation*}

	\end{enumerate}
\end{hypo}
The conditions \ref{hyp:coef_cont} to \ref{hyp:init_indep} guarantee that \cref{eqn:sde_y} with the initial condition \(y_0 = x\) has a unique strong solution \citep{KallianpurSundar_2014_StochasticAnalysisDiffusion}.
The bound \(K_{\nabla\nabla u}\) placed on the second derivatives of \(u\) in \ref{hyp:nnu_bounds} quantifies exactly when the deterministic dynamics (that is, \(u\)) of \cref{eqn:sde_y} are linear.
Similarly, the bound \(K_{\nabla\sigma}\) on the spatial derivatives of \(\sigma\) in \ref{hyp:sigma_deriv_bound} allows us to distinguish when the noise in \cref{eqn:sde_y} is multiplicative.

Our aim is to construct and formally justify a computable linearisation of \cref{eqn:sde_y} about a trajectory solving the deterministic system \cref{eqn:ode_det}.
To that end, we take a \emph{fixed} initial condition \(x_0 \in \R^n\) to the reference deterministic model \cref{eqn:ode_det} and consider linearising the SDE \cref{eqn:sde_y} about the corresponding trajectory \(F_0^t\!\left(x_0\right)\).
We consider the following linearisation of \cref{eqn:sde_y}:
\begin{equation}
	\dif l_t^{(\epsilon)} = \left[u\!\left(F_0^t\!\left(x_0\right), t\right) + \nabla u\!\left(F_0^t\!\left(x_0\right), t\right)\left(l_t^{(\epsilon)} - F_0^t\!\left(x_0\right)\right)\right] \! \dif t + \epsilon\sigma\!\left(F_0^t\!\left(x_0\right), t\right) \! \dif W_t, \quad l_0^{(\epsilon)} = x,
	\label{eqn:linear_sde_inform}
\end{equation}
where the initial condition \(x \) is still permitted to be random.
Informally, we can arrive at \cref{eqn:linear_sde_inform} by performing a Taylor expansion of the coefficient \(u\) up to first-order and \(\sigma\) to zeroth-order about the time-varying trajectory \(F_0^t\!\left(x_0\right)\).
Such a linearisation is advantageous over the nonlinear SDE \cref{eqn:sde_y}, since \cref{eqn:linear_sde_inform} can be solved analytically.
We will later (see \Cref{cor:limit_moments}) provide explicit expressions for computing the distribution of the solution \(l_t^{(\epsilon)}\) solely in terms of the solution dynamics of the deterministic system \cref{eqn:ode_det}, the specified diffusion matrix \(\sigma\), and the distribution of \(x\).


In order to quantify the error arising from the choice of reference point \(x_0\), we define
\[
	\delta_r \coloneqq \avg{\norm{x - x_0}^r}^{1/r},
\]
i.e.\ \(\delta_r\) is the \(L_r\) distance between \(x\) and the deterministic point \(x_0\).
We can think of \(\delta_r\) as a scalar measure of the uncertainty in the initial condition, relative to the choice of reference point \(x_0\).
Alternatively, the limit as \(\delta_r\) approaches zero is equivalent to convergence in \(r\)th mean of \(x\) to the fixed point \(x_0\).
We can therefore distinguish two sources of uncertainty in our model; that arising from the initial condition, quantified by \(\delta_r\), and the ongoing uncertainty driven by the Wiener process \(W_t\) as measured by \(\epsilon\).

Our first and primary result, \Cref{thm:main}, provides an explicit bound on the \(r\)th moment of the error between the SDE solution \(y_t^{(\epsilon)}\) and the linearised solution \(l_t^{(\epsilon)}\).


\begin{theorem}[Linearisation error is bounded]\label{thm:main}
	Let \(y_t^{(\epsilon)}\) be the strong solution to the SDE \cref{eqn:sde_y} and \(l_t^{(\epsilon)}\) be the strong solution to the corresponding linearisation \cref{eqn:linear_sde_inform}, both driven by the same Wiener process \(W_t\) and subject to the same random initial condition \(y_0^{(\epsilon)} = l_0^{(\epsilon)} = x\).
	Then, for any \(r \geq 1\) such that \(\delta_{2r} < \infty\) and \(t \in [0,T]\), there exist constants \( D_1\!\left(r,t, K_{\nabla u}, K_{\sigma}\right), \, D_2\!\left(r,t, K_{\nabla u}\right), \, D_3\!\left(r,t, K_{\nabla u}\right) \in [0, \infty) \) independent of \(x\) and \(x_0\) such that for all \(\epsilon > 0\),
	\begin{equation}
		\avg{\norm{y_t^{(\epsilon)} - l_t^{(\epsilon)}}^r} \leq \begin{multlined}[t]
			\left(K_{\nabla\nabla u}^r + K_{\nabla\sigma}^r\right) D_1\!\left(r,t, K_{\nabla u}, K_\sigma\right)\, \epsilon^{2r} \\
			+ K_{\nabla\nabla u}^r D_2\!\left(r,t, K_{\nabla u}\right)\delta_{2r}^{2r}
			+ K_{\nabla\sigma}^r D_3\!\left(r,t, K_{\nabla u}\right)\delta_r^r \epsilon^r.
		\end{multlined}
		\label{eqn:main_ineq}
	\end{equation}
\end{theorem}

\begin{proof}
	See \Cref{app:main_thm_proof}.
	Our proof employs the Burkholder-Davis-Gundy inequality, Gr\"onwall's inequality, and Taylor's theorem to explicitly construct the bounding coefficients in terms of the conditions on the SDE coefficients set out in \Cref{hyp:smooth}.
	The bounding coefficients \(D_1\), \(D_2\), and \(D_3\) are given explicitly in \cref{eqn:bound_defns}.
\end{proof}


In \cref{eqn:main_ineq}, we have an explicit scaling of the error in terms of \(\epsilon\) and \(\delta_r\).
The three terms can be informally considered as: a contribution purely from the ongoing linearisation error, a contribution purely from the initial uncertainty, and a term resulting from the interaction between the initial and ongoing uncertainties.
By explicitly identifying the dependence of the bound on \(K_{\nabla\nabla u}\) and \(K_{\nabla \sigma}\), we note three special cases that are summarised by \Crefrange{rem:bound_linear}{rem:bound_exact}.

\begin{remark}[Linear drift]\label{rem:bound_linear}
	When the deterministic dynamics are linear, we can set \(K_{\nabla\nabla u} = 0\) and \cref{eqn:main_ineq} becomes
	\[
		\avg{\norm{y_t^{(\epsilon)} - l_t^{(\epsilon)}}^r} \leq   K_{\nabla\sigma}^r D_1\!\left(r, t, K_{\nabla u}, K_\sigma\right)\, \epsilon^{2r} + K_{\nabla\sigma}^r D_3\!\left(r,t, K_{\nabla u}\right)\delta_r^r \epsilon^r.
	\]
	The linearisation of the drift term \(u\) is exact, so the error is purely due to the spatial dependency of the diffusion term \(\sigma\).
\end{remark}

\begin{remark}[Additive noise]\label{rem:bound_additive}
	When the noise in \cref{eqn:sde_y} is additive, we can set \(K_{\nabla\sigma} = 0\) and \cref{eqn:main_ineq} becomes
	\[
		\avg{\norm{y_t^{(\epsilon)} - l_t^{(\epsilon)}}^r} \leq   K_{\nabla\nabla u}^r D_1\!\left(r,t, K_{\nabla u}, K_\sigma\right)\, \epsilon^{2r} + K_{\nabla\nabla u}^r D_2\!\left(r,t, K_{\nabla u}\right)\delta_{2r}^{2r}.
	\]
	The error is then purely due to the linearisation of the drift term \(u\), and as expected is of second order in both the initial condition uncertainty \(\delta_{2r}\) and the ongoing uncertainty \(\epsilon\).
\end{remark}

\begin{remark}[Exact linearisation]\label{rem:bound_exact}
	When the deterministic dynamics are linear and the noise in \cref{eqn:sde_y} is additive (non-multiplicative), the linearisation \cref{eqn:linear_sde_inform} should be exact.
	Accordingly, we can set \(K_{\nabla\nabla u} = K_{\nabla\sigma} = 0\) and \cref{eqn:main_ineq} becomes
	\[
		\avg{\norm{y_t^{(\epsilon)} - l_t^{(\epsilon)}}^r} = 0.
	\]
	In turn, this implies that \(y_t^{(\epsilon)} = l_t^{(\epsilon)}\) almost surely, for any choice of reference point \(x_0\).
\end{remark}


We postpone a discussion of an additional special case---where the initial condition is fixed or Gaussian---to a later section.
For the general situation, we next explicitly establish the solution to the linearisation \cref{eqn:linear_sde_inform}, in terms of the initial condition and the deterministic evolution of \cref{eqn:ode_det}.

\begin{theorem}[Solution of the linearised SDE]\label{thm:limit_sol}
	The linearised SDE \cref{eqn:linear_sde_inform} has the strong solution
	\begin{equation}
		l_t^{(\epsilon)} = \nabla F_0^t\!\left(x_0\right)\left(x - x_0\right) + F_0^t\!\left(x_0\right) + \epsilon\nabla F_0^t\!\left(x_0\right)\int_0^t{L\!\left(x_0, \tau\right)\dif W_\tau}.
		\label{eqn:linear_sol}
	\end{equation}
	where the term involving the uncertain initial condition \(x\) and the It\^o integral are independent, and
	\begin{equation}
		L\!\left(x_0, \tau\right) \coloneqq \left[\nabla F_0^\tau(x_0)\right]^{-1}\sigma\left(F_0^\tau(x_0), \tau\right).
		\label{eqn:sigma_L_def}
	\end{equation}

\end{theorem}
\begin{proof}
	See \Cref{app:limit_sol_proof}.
\end{proof}

The representation of the linearised solution as an independent sum in \cref{eqn:linear_sol} can be seen as a decomposition into contributions from the initial uncertainty (the transformation of initial condition \(x\)), a deterministic prediction (the flow map \(F_0^t\!\left(x_0\right)\)) and the ongoing uncertainty in \(u\) (the remaining It\^o integral term).

We can further show that the It\^o integral term follows a Gaussian random variable, which ensures that the independent sum in \cref{eqn:linear_sol} is a convenient expression for both theoretical analysis and numerical computation.
We also provide explicit expressions for the mean and covariance matrix of the linearised solution, written in terms of the deterministic dynamics and \(\sigma\).
\begin{corollary}[Distribution of the linearised solution]\label{cor:limit_moments}
	The It\^o integral term in \cref{eqn:linear_sol} follows a Gaussian distribution independent of \(x\), namely
	\[
		\int_0^t{L\!\left(x_0,\tau\right)\dif W_\tau} \isGauss{0, \int_0^t{L\!\left(x_0, \tau\right)L\!\left(x_0, \tau\right)^{\T}\dif\tau}}.
	\]
	The mean of the linearised solution is
	\begin{equation}
		\avg{l_t^{(\epsilon)}} = F_0^t\!\left(x_0\right) + \nabla F_0^t\!\left(x_0\right) \avg{x - x_0}.
		\label{eqn:mean_expl_eqn}
	\end{equation}
	The \(n\times n\) covariance matrix of the linearised solution is given explicitly by
	\begin{equation}
		\var{l_t^{(\epsilon)}} = \nabla F_0^t\!\left(x_0\right)\left(\var{x} + \epsilon^2 \int_0^t{L\!\left(x_0,\tau\right)L\!\left(x_0,\tau\right)^{\T}\dif\tau}\right)\left[\nabla F_0^t\!\left(x_0\right)\right]^{\T}
		\label{eqn:pi_expl_eqn}
	\end{equation}
	where \(L\!\left(x_0, \tau\right)\) is as defined in \cref{eqn:sigma_L_def} and the integral is taken in the elementwise sense.
\end{corollary}
\begin{proof}
	See \Cref{app:limit_moments_proof}.
	The expressions follow from the representation of the linearised solution as an independent sum in \cref{eqn:linear_sol}.
\end{proof}

In \Cref{thm:limit_sol}, we have provided expressions for the distribution of the solution \(l_t^{(\epsilon)}\) to the linearised SDE \cref{eqn:linear_sde_inform} written solely in terms of the behaviour of the deterministic system \cref{eqn:ode_det}, the specified diffusion matrix \(\sigma\), and the distribution of the initial condition \(x\).
This describes a method for approximating the solution to the nonlinear SDE \cref{eqn:sde_y}, or for characterising the impact of uncertainty in a dynamical system \cref{eqn:ode_det}, that circumvents the need for expensive stochastic simulation.

Thus far, we have stated our results in terms of a general initial condition \(x\), and provided expressions for the linearised solution in terms of this otherwise arbitrary distribution.
However, we later consider two special cases for the initial condition \(x\), a fixed (deterministic) initial condition in \Cref{sec:theory_fixed}, and a Gaussian initial condition in \Cref{sec:theory_gauss}.
In both these cases, the linearised solution also follows a Gaussian distribution which is characterised entirely by the mean and covariance described in \Cref{cor:limit_moments}, allowing for easy computation.
We also relate these results directly to other literature \citep{Jazwinski_2014_StochasticProcessesFiltering,FreidlinWentzell_1998_RandomPerturbationsDynamical,Blagoveshchenskii_1962_DiffusionProcessesDepending,Balasuriya_2020_StochasticSensitivityComputable,Sanz-AlonsoStuart_2017_GaussianApproximationsSmall,SarkkaSolin_2019_AppliedStochasticDifferential} which uses linearisation procedures and Gaussian process approximations for nonlinear SDEs in these situations.

Next, we establish the ordinary differential equation satisfied by the covariance matrix, which is an expression consistent with linearisations schemes described elsewhere \citep{ArchambeauEtAl_2007_GaussianProcessApproximations,SarkkaSolin_2019_AppliedStochasticDifferential,Jazwinski_2014_StochasticProcessesFiltering,Sanz-AlonsoStuart_2017_GaussianApproximationsSmall}.
This ODE enables rapid computation of the mean and covariance of the linearised solutions by solving a system of ODEs, i.e.\ \cref{eqn:ode_det} and \cref{eqn:pi_ode}.

\begin{remark}\label{rem:cov_ode}
	The \(n\times n\) covariance matrix \(\var{l_t^{(\epsilon)}}\) of the linearised solution is the symmetric positive-semidefinite \(n \times n\) matrix solution to the ordinary differential equation
	\begin{equation}
		\dod{\Pi(t)}{t} = \begin{multlined}[t]
			\nabla u\!\left(F_0^t\!\left(x_0\right), t\right) \Pi(t) + \Pi(t)\left[\nabla u\!\left(F_0^t\!\left(x_0\right), t\right)\right]^{\!\T}\! + \epsilon^2\sigma\!\left(F_0^t\!\left(x_0\right), t\right)\sigma\!\left(F_0^t\!\left(x_0\right), t\right)^{\T}\!,
		\end{multlined}
		\label{eqn:pi_ode}
	\end{equation}
	subject to the initial condition \(\Pi(0) = \var{x}\).
	We show that the variance satisfies \cref{eqn:pi_ode} in \Cref{app:limit_moments_proof}.
\end{remark}





%%%%%%%%%%%%%%%%%%%%%%%%%%%%%%%%%%%%%%%%%%%%%%%%%%%%
\subsection{Comparison to existing results}
\label{sec:comparison}
In this section we connect our work to the cognate bound derived by \citet{Sanz-AlonsoStuart_2017_GaussianApproximationsSmall}.
That paper considers the following SDE:
\begin{equation}\label{eqn:SAS_sde}
	\dif y_t^{(\epsilon)} = u\!\left(y_t^{(\epsilon)}\right)\dif t + \epsilon \, \tilde{\sigma}\dif W_t,
\end{equation}
where the diffusion coefficient \(\tilde{\sigma}\) is a constant matrix, which is a special case of \cref{eqn:sde_y}.
In this section, we apply our results to \cref{eqn:SAS_sde} to enable both bounds to be compared.
Note that \(\epsilon\) in our article is written as \(\sqrt{\epsilon}\) in \citet{Sanz-AlonsoStuart_2017_GaussianApproximationsSmall}; we will translate results from \citet{Sanz-AlonsoStuart_2017_GaussianApproximationsSmall} to use our notation, so that all results in this article are directly comparable.
In the following, \(c\) denotes an arbitrary finite and non-negative constant that can vary between inequalities.

Theorem 2.2 of \citet{Sanz-AlonsoStuart_2017_GaussianApproximationsSmall}, summarised, is as follows.
Let \(\xi_t^{(\epsilon)}\) be the probability measure associated with \(y_t^{(\epsilon)}\) (as defined in \cref{eqn:SAS_sde}), and let \(\nu_t^{(\epsilon)}\) be the probability measure associated with the corresponding linearisation \(l_t^{(\epsilon)}\) (as defined in \cref{sec:theory}).
Then there exists a constant \(c\) such that the Kullback--Leibler (KL) divergence \(D_{\mathrm{KL}}\) between \(\xi_t^{(\epsilon)}\) and \(\nu_t^{(\epsilon)}\) is bounded;
\begin{equation}
	\label{eqn:SAS}
	D_{\mathrm{KL}}\!\left(\xi_t^{(\epsilon)} \,\middle|\middle|\, \nu_t^{(\epsilon)}\right) \le D_{\mathrm{KL}}\!\left(\xi_0^{(\epsilon)} \,\middle|\middle|\, \nu_0^{(\epsilon)}\right) + c \, \epsilon^2\;.
\end{equation}
To focus on the scaling with \(\epsilon\), assume a fixed initial condition with \(D_{\mathrm{KL}}\!\left(\xi_0^{(\epsilon)} \,\middle|\middle|\, \nu_0^{(\epsilon)}\right) = 0\) (and \(\delta_r = 0\) in our bound \cref{eqn:main_ineq}). Then, employing the Hellinger distance \(D_{\mathrm{H}}\), \cref{eqn:SAS} implies
\begin{align}
	\label{eqn:convert}
	\norm{\avg{ y_t^{(\epsilon)} - l_t^{(\epsilon)}} } \le c D_{\mathrm{H}}\!\left(\xi_t^{(\epsilon)} ,\, \nu_t^{(\epsilon)}\right) \le c \sqrt{D_{\mathrm{KL}}\!\left(\xi_t^{(\epsilon)} \,\middle|\middle|\, \nu_t^{(\epsilon)}\right)} \le c \, \epsilon \;,
\end{align}
while our result \cref{eqn:main_ineq} and Jensen's inequality imply
\[
	\norm{\avg{ y_t^{(\epsilon)} - l_t^{(\epsilon)}} } \le \avg{\norm{y_t^{(\epsilon)} - l_t^{(\epsilon)}}} \leq c \, \epsilon^2\;.
\]
Thus, our bound on the moments is quadratic in \( \epsilon \) rather than linear.
If our conversion in \cref{eqn:convert} was optimal, then our approach in this article
provides a sharper bound on \(\norm{\avg{y_t^{(\epsilon)} - l_t^{(\epsilon)}}}\) that the
results of  \citet{Sanz-AlonsoStuart_2017_GaussianApproximationsSmall} imply, and do so for a
more general $ \sigma $.
The results in  \citet{Sanz-AlonsoStuart_2017_GaussianApproximationsSmall} on the KL divergence would be more natural in information-theoretic contexts, and our hope is that our explicit bound on the moments would be similarly preferred in other contexts.




\subsection{Gaussian initial condition}\label{sec:theory_gauss}
We now briefly consider the case when the initial condition follows a Gaussian distribution, i.e.\ \(x \isGauss{\mu_0, \Sigma_0}\), where \(\mu_0 \in \R^n\) and \(\Sigma_0 \in \R^{n \times n}\) are fixed and specified.
The linearisation then follows a Gaussian distribution itself, which is entirely characterised by the mean and covariance matrix described in \Cref{cor:limit_moments}.
Alternatively, these moments can be conveniently computed by simultaneously solving \cref{eqn:ode_det} for the state variable and \cref{eqn:pi_ode} for the linearised covariance.

A natural choice of reference point \(x_0\) is the mean of the initial Gaussian density, i.e.\ \(x_0 = \mu_0\).
The \(L_r\) distance between \(x\) and the mean \(\mu_0\) can be bounded by the trace of \(\Sigma_0\); for example, one such bound is
\begin{equation}\label{eqn:gauss_dist_bound}
	\delta_r^{r} \leq n^{3r/2 - 1} M_r \mathrm{tr}\left(\Sigma_0\right)^{r/2}, \quad M_r \coloneqq \frac{2^{r/2}\Gamma\!\left(\frac{r + 1}{2}\right)}{\sqrt{\pi}},
\end{equation}
where \(\Gamma\) denotes the Gamma function, with equality when \(n = 1\).
The initial covariance \(\Sigma_0\) directly measures the uncertainty in the initial condition, and we see through \cref{eqn:gauss_dist_bound} that as the components of \(\Sigma_0\) approach zero, the contribution of the initial uncertainty to the linearisation error in \cref{eqn:main_ineq} approaches zero also.
The linearised solution is then
\[
	l_t^{(\epsilon)} \isGauss{F_0^t\!\left(x_0\right), \, \nabla F_0^t\!\left(x_0\right) \Sigma_0\left[\nabla F_0^t\!\left(x_0\right)\right]^{\T} + \varepsilon^2 \Sigma_0^t\!\left(x_0\right)},
\]
where \(\Sigma_0^t\!\left(x_0\right)\) is given explicitly by
\begin{equation}\label{eqn:sigma_def}
	\Sigma_0^t\!\left(x_0\right) = \nabla F_0^t\!\left(x_0\right)\left(\int_0^t{L\!\left(x_0, \tau\right)L\!\left(x_0, \tau\right)^{\T} \dif\tau}\right)\left[\nabla F_0^t\!\left(x_0\right)\right]^{\T},
\end{equation}
and is the solution to the matrix differential equation \cref{eqn:pi_ode} in \Cref{rem:cov_ode}, subject to \(\Sigma_0^0\!\left(x_0\right) = O\), the \(n \times n\) zero matrix.
The covariance matrix \(\Sigma_0^t\!\left(x_0\right)\) characterises the contribution of the ongoing uncertainty in the stochastic system.
The full covariance matrix \(\var{l_t^{(\epsilon)}}\) is also the solution to \cref{eqn:pi_ode} subject to the initial condition \(\Pi(0) = \var{x}\).
By jointly solving \cref{eqn:ode_det} for the deterministic trajectory (the mean of \(l_t^{(\epsilon)}\)) and \cref{eqn:pi_ode} for the covariance matrix, one can easily compute the linearised solution, describing exactly the assumed Gaussian approximation presented in \citet{SarkkaSolin_2019_AppliedStochasticDifferential}, and the dynamics linearisation used in the extended Kalman filter \citep{Jazwinski_2014_StochasticProcessesFiltering}.


%%%%%%%%%%%%%%%%%%%%%%%%%%%%%%%%%%%%%%%%%%%%%%%%%%%%
\subsection{Fixed initial condition}\label{sec:theory_fixed}
Consider when the initial condition \(x\) is itself a fixed and known deterministic value, in which case we take \(x = x_0\) and \(\delta_r = 0\) for all \(r\).
In this situation, the bound \cref{eqn:main_ineq} on the linearisation error reduces to
\begin{equation}
	\avg{\norm{y_t^{(\epsilon)} - l_t^{(\epsilon)}}^r} \leq \left(K_{\nabla\nabla u}^r + K_{\nabla\sigma}^r\right)D_1\!\left(r,t, K_{\nabla u}, K_\sigma\right)\epsilon^{2r}.
	\label{eqn:main_ineq_fixed}
\end{equation}
We can consider the linearisation as equivalently arising from a first-order power series expansion of \(y_t^{(\epsilon)}\) in the noise-scale parameter \(\epsilon\), i.e.
\[
	y_t^{(\epsilon)} = F_0^t\!\left(x_0\right) + \epsilon z_t^{(\epsilon)} + R_2\left(x,t,\epsilon\right).
\]
where \(z_\epsilon \coloneqq \left(l_{t}^{(\epsilon)} - F_0^t\!\left(x_0\right)\right) / \epsilon\) is the first order term and \(R_2\) is a random quantity capturing the remaining deviation between \(y_t^{(\epsilon)}\) and the linearisation.
By rearranging and taking \(r = 1\) in \cref{eqn:main_ineq_fixed}, we therefore have the explicit Taylor-like bound
\[
	\frac{\avg{\norm{R_2\left(x,t,\epsilon\right)}}}{\epsilon^2} \leq \left(K_{\nabla \nabla u} + K_{\nabla\sigma}\right)D_1\!\left(1,t, K_{\nabla u}, K_\sigma\right),
\]
This result is consistent in formulation with the error bounds of \citet{Blagoveshchenskii_1962_DiffusionProcessesDepending} and \citet{FreidlinWentzell_1998_RandomPerturbationsDynamical}, for instance.
Moreover, the distribution of the linearisation solution \cref{eqn:linear_sol} is Gaussian, which through \Cref{cor:limit_moments} we can again explicitly characterise in terms of the deterministic system, namely
\begin{equation}
	l_t^{(\epsilon)} \isGauss{F_0^t\!\left(x_0\right), \epsilon^2 \Sigma_0^t\!\left(x_0\right)},
	\label{eqn:linear_gauss_sol}
\end{equation}
where \(\Sigma_0^t\!\left(x_0\right)\) is defined in \cref{eqn:sigma_def}.
The distribution can be computed \emph{entirely} from the solution behaviour of the deterministic equation \cref{eqn:ode_det} and prior specification of \(\sigma\).
In \Cref{sec:theory_s2}, we demonstrate an application of these results to extend stochastic sensitivity \citep{Balasuriya_2020_StochasticSensitivityComputable} to arbitrary dimension.


%%%%%%%%%%%%%%%%%%%%%%%%%%%%%%%%%%%%%%%%%%%%%%%%%%%%
\section{Extending stochastic sensitivity}\label{sec:theory_s2}
The results of \Cref{sec:theory_fixed} for a fixed initial condition provide a direct extension of the stochastic sensitivity tools first introduced by \citet{Balasuriya_2020_StochasticSensitivityComputable} for the fluid flow context.
Here, the deterministic model \cref{eqn:ode_det} is seen as a ``best-available'' model for the evolution of Lagrangian trajectories, and the driving vector field \(u\) is the Eulerian velocity of the fluid.
Stochastic sensitivity ascribes a scalar value to each deterministic trajectory by computing a maximum variance of projected deviation \citep{Balasuriya_2020_StochasticSensitivityComputable}.
The aim is to provide a \emph{single} computable number for each deterministic trajectory quantifying the impact of uncertainty in the velocity, independent of the scale (\(\epsilon\)) of the noise.
The natural restating of this original definition of stochastic sensitivity \citep{Balasuriya_2020_StochasticSensitivityComputable} in the $ n $-dimensional setting is as follows:

\begin{figure}
	\centering
	\begin{tikzpicture}[scale=2]
		\pgfmathsetseed{1}

		% Actual path
		\draw (1,1) .. controls (2.5,-0.5) and (4,1.4) .. (5, 0.5);

		% Stochastic path -- just place a random walk between specified points
		\draw[color=ForestGreen, decorate, decoration={random steps, segment length = 1pt, amplitude=2pt}] (1,1) .. controls (2.5, -0.2) and (3.7,2.7) .. (6.4, 2.1);

		% S2 quantities
		\path[-{Latex[length=3mm, width=1.5mm]},blue] (5,0.5) edge node[left, xshift=-5pt] {\(y_t^{(\epsilon)} - F_0^t\!\left(x_0\right)\)} (6.4, 2.1);
		% Calculation here: fix theta, the angle of the projection. Length of the vector between det point and end of projection is l*sin(n), where l is the length of the vector between the det and stoch points, and n = pi/2 - phi + theta, where phi is the angle of the vector between the det and stoch points.
		\def\vecl{sqrt(1.4^2 + 1.6^2)};
		\def\projtheta{pi/7};
		\def\projphi{atan(1.6 / 1.4) * pi / 180}
		\path[red] (5,0.5) edge ({5 +\vecl*cos((\projphi - \projtheta) r) * cos(\projtheta r)}, {0.5 + \vecl* cos((\projphi - \projtheta) r) * sin(\projtheta r)});

		\path[dashed, gray] (6.4, 2.1) edge ({5 + \vecl * cos((\projphi - \projtheta) r) * cos(\projtheta r)}, {0.5 + \vecl * cos((\projphi - \projtheta) r) * sin(\projtheta r)});

		% Reference angle
		% \path[dashed,gray] (5,0.5) edge (7,0.5);
		% \draw[gray] (5.5,0.5) arc (0:{\projtheta * 180 / pi}:0.5) node[midway, xshift=5pt, yshift=2pt] {\(\theta\)};

		% Instead, we get a vector p
		\path[-{Latex[length=3mm, width=1.5mm]}, gray] (5,0.5) edge node[right, xshift=5pt, yshift=-2pt] {\(p\)} ({5 + cos(\projtheta r)}, {0.5 + sin(\projtheta r)});


		% Points
		\filldraw [black] (1,1) circle (1pt) node[anchor=east] {\(x_0\)};
		\filldraw [black] (5,0.5) circle (1pt) node [anchor=north] {\(F_0^t\!\left(x_0\right)\)};
		\filldraw [color=ForestGreen] (6.4, 2.1) circle (1pt) node [anchor=south] {\(y_t^{(\epsilon)}\)};

	\end{tikzpicture}
	\caption{The entities used in \Cref{def:ss_Rn} of stochastic sensitivity in arbitrary dimensions.
		This diagram should be compared to \Cref{fig:s2_diag}.
	}
	\label{fig:s2_diag_Rn}
\end{figure}

\begin{definition}[Stochastic sensitivity in \(\R^n\)]\label{def:ss_Rn}
	The \emph{stochastic sensitivity} is a scalar field \(S^2\colon \R^n \times [0,T] \to \left[0, \infty\right)\) given by
	\begin{equation*}
		S^2\!\left(x_0,t\right) \coloneqq \lim_{\epsilon\downarrow 0}\sup\set{\var{\frac{1}{\epsilon}p^{\T}\left(y_t^{(\epsilon)} - F_0^t\!\left(x_0\right)\right)} \,: \, p \in \R^n, \, \norm{p} = 1}.
	\end{equation*}
\end{definition}

\Cref{fig:s2_diag_Rn} illustrates the quantities involved in \Cref{def:ss_Rn}, to be compared directly to \Cref{fig:s2_diag} which represented the original definition of stochastic sensitivity.
The ray of angle \(\theta\) in the original definition has been replaced with an arbitrary unit vector \(p\).
We compute the projection (in red) of the deviation (in blue) between the deterministic (in black) and stochastic (in green) predictions and then maximise over all possible unit vectors \(p\) to obtain the stochastic sensitivity value.
\Cref{def:ss_Rn} is in the spirit of principal components analysis \citep{Jolliffe_2002_PrincipalComponentAnalysis}, performing a dimension reduction by projecting onto the direction in which the variance is maximised, thus capturing the most uncertainty in the data with a scalar value.
The anisotropic uncertainty in two-dimensions \citep{Balasuriya_2020_StochasticSensitivityComputable} is the direction-dependent projection (before optimising over all directions in \Cref{def:ss_Rn}).
Explicit theoretical expressions for both the stochastic sensitivity and the anisotropic sensitivity in two dimensions were obtained by \citet{Balasuriya_2020_StochasticSensitivityComputable}; these allowed for quantifying certainty in eventual trajectory locations without having to perform stochastic simulations.
We show here that our results in \(n\)-dimensions are a generalisation of the two-dimensional ones by \citet{Balasuriya_2020_StochasticSensitivityComputable}, which moreover establishes Gaussianity as well as an explicit expression for the uncertainty measure.
A theoretically pleasing and computable expression for the stochastic sensitivity is obtainable:


\begin{theorem}[Computation of \(S^2\)]\label{thm:s2_calculation}
	For any \(x_0 \in \R^n\) and \(t \in [0,T]\),
	\begin{equation}
		S^2\!\left(x_0,t\right) = \norm{\Sigma_0^t\!\left(x_0\right)},
		\label{eqn:s2_calculation}
	\end{equation}
	where the covariance matrix \(\Sigma_0^t\) is defined in \cref{eqn:sigma_def}.
	Equivalently, \(S^2\!\left(x_0,t\right)\) is given by the maximum eigenvalue of \(\Sigma_0^t\!\left(x_0\right)\).
\end{theorem}
\begin{proof}
	See \Cref{app:s2_calculation_proof}.
	This result uses \Cref{thm:main} to establish the convergence of the covariance matrices, and then the properties of the spectral norm to establish \cref{eqn:s2_calculation}.
\end{proof}

Independent of the fluid mechanics context, \Cref{thm:s2_calculation} indicates that even for general systems, the matrix norm of \(\Sigma_0^t(x)\), i.e.\ the stochastic sensitivity \(S^2(x,t)\), can be used as {\em one} number which encapsulates the uncertainty of an initial state \(x\) after \(t\) time units.
The significance of this result is that the stochastic sensitivity has here been recovered as the maximal eigenvalue of a covariance matrix that is ubiquitous in the literature.
Stochastic sensitivity was formerly defined by \citet{Balasuriya_2020_StochasticSensitivityComputable} as the maximal value of the anisotropic uncertainty of a particular stochastic flow, and the connection to a linearisation was not apparent.

The stochastic sensitivity field can be calculated given any velocity data \(u\), and through the explicit expression \cref{eqn:sigma_def} for \(\Sigma_0^t\) can even be computed from only flow map data.
Computation does not require knowledge of the noise scale \(\epsilon\), so the \(S^2\) field is intrinsic in capturing the impact of the model dynamics on uncertainty, and any specified non-uniform diffusivity.
It has already been shown that, in the fluid flow context, stochastic sensitivity can identify coherent regions in two-dimensions \citep{BadzaEtAl_2023_HowSensitiveAre, Balasuriya_2020_StochasticSensitivityComputable}.
A simple approach is to define robust sets, which are those initial conditions for which the corresponding \(S^2\) value, i.e.\ the uncertainty in eventual location, are below some specified threshold.
This threshold can be defined precisely in terms of a spatial lengthscale of interest and the advective and diffusive characteristics of the flow, as Definition 2.9 of \citet{Balasuriya_2020_StochasticSensitivityComputable}.
Such a definition extends to the \(n\)-dimensional case as presented here, moreover establishing an easily computable method for determining coherent sets from the covariance matrix \(\Sigma_0^t\).

% \begin{definition}[Robust sets in \(\R^n\)]\label{def:ss_robust}
% 	Given a threshold \(R_0 > 0\), the set
% 	\[
% 		R\!\left(R_0, t\right) \coloneqq \setc{x_0 \in \R^n}{S^2\!\left(x_0, t\right) < R},
% 	\]
% \end{definition}



\subsection{Connections to the finite-time Lyapunov exponent}\label{sec:ftle_s2_connection}
We briefly consider the implications of our results on uncertain initial conditions from \Cref{sec:theory} on stochastic sensitivity, which suggests a connection between this measure and the finite-time Lyapunov exponent.
The finite-time Lyapunov exponent (FTLE) is a measure of the local stretching in a dynamical system and is one of the most commonly used tools for identifying Lagrangian coherent structures in fluid flow \citep{ShaddenEtAl_2005_DefinitionPropertiesLagrangian}.
In a particular direction \(p\) from the initial condition \(x_0\), the stretching due to the evolution of the ODE \cref{eqn:det_ode} is measured as
\[
	s\!\left(x_0,p\right) = \frac{\norm{F_0^t\!\left(x_0 + p\right) - F_0^t\!\left(x_0\right)}}{\norm{x_0}} \approx \frac{\norm{F_0^t\!\left(x_0\right)p}}{\norm{p}}
\]
assuming the perturbation \(p\) is small so that \(\mathcal{O}\!\left(\norm{p}^2\right)\) terms can be neglected.
The FTLE is then computed by finding the direction in which the stretching is maximised, leading to the operator norm computation
\[
	\sup_{p \in \R^n, \, p \neq 0} s\!\left(x, p\right) = \norm{\nabla F_0^t\!\left(x_0\right)} = \sqrt{\norm{\left[\nabla F_0^t\!\left(x_0\right)\right]^{\T}\nabla F_0^t\!\left(x_0\right)}}.
\]
The stochastic sensitivity definition in \Cref{def:ss_Rn} is analogous to this formation of the FTLE.
We consider a small stochastic perturbation (the solution to the SDE \cref{eqn:sde_y}) to a deterministic model (the corresponding deterministic system \cref{eqn:det_ode}).
The size of the perturbation is parameterised by \(\epsilon\), and the notion that this perturbation is `small' is formalised by considering the limit as \(\epsilon\) approaches zero.
Since we are now dealing with stochastic quantities, we take the direction in which the variance of the mapped perturbation is maximised.
\Cref{thm:s2_calculation} further establishes that stochastic sensitivity can be computed by taking the operator norm of the covariance matrix of the linearised SDE with a fixed initial condition.
That is, stochastic sensitivity is the operator norm of
\[
	\frac{1}{\epsilon^2}\var{l_t^{(\epsilon)}} = \Sigma_0^t\!\left(x_0\right).
\]
This matrix captures (up to leading order in \(\epsilon\)) the impact of the ongoing uncertainty.
However, the theory of \Cref{sec:theory} allows for uncertain initial conditions, so suppose that we instead consider the perturbation SDE \cref{eqn:sde_y} initial condition \(x\) with expectation \(\avg{x} = x_0\) and variance \(\var{x} = \epsilon^2 I\).
If there was no ongoing uncertainty (\(\sigma \equiv O\)), then any uncertainty in the system arises purely from the initial condition \(x\).
Then from \cref{eqn:pi_expl_eqn}, the variance of the small noise linearisation is
\[
	\frac{1}{\epsilon^2}\var{l_t^{(\epsilon)}} = \nabla F_0^t\!\left(x_0\right)\left[\nabla F_0^t\!\left(x_0\right)\right]^{\T},
\]
resulting only from the first term in \cref{eqn:pi_expl_eqn}.
Thus, if we compute stochastic sensitivity in the same way as in \Cref{def:ss_Rn} and \Cref{thm:s2_calculation} we get
\[
	\tilde{S}^2\!\left(x_0, t\right) = \norm{\nabla F_0^t\!\left(x_0\right)\left[\nabla F_0^t\!\left(x_0\right)\right]^{\T}} = \sup_{p \in \R^n, \, p \neq 0}{s\!\left(x_0, p\right)^2}
\]
By maximising the projection of this variance over all directions, as in \cref{eqn:s2_calculation} to compute stochastic sensitivity, we perform exactly the computation to determine the maximal stretching rate in the computation of the finite-time Lyapunov exponent.
The finite-time Lyapunov exponent (FTLE) quantifies the sensitivity of a dynamical system to initial conditions \citep{ShaddenEtAl_2005_DefinitionPropertiesLagrangian}, and can be equivalently considered a measure of the impact of an uncertain initial condition on the \emph{deterministic} evolution of trajectories.
The construction of \(\tilde{S}^2\) is completely consistent with this idea.
Thus, while the original formulation of stochastic sensitivity with a fixed initial condition captures only the impact of ongoing uncertainty, by including an uncertain initial condition in the computation we can also account for the information on the impact of initial condition uncertainty provided by the FTLE.
There has been recent interest in extending the FTLE for systems with ongoing uncertainty \citep{Balasuriya_2020_UncertaintyFinitetimeLyapunov,YouLeung_2021_ComputingFiniteTime,GuoEtAl_2016_FiniteTimeLyapunovExponents}, but no established approach as of yet.
A framework that computes stochastic sensitivity with uncertain initial conditions can be seen as such an extension of the FTLE, in the sense that the measure would characterise the sensitivity of a dynamical system to \emph{both} initial conditions and ongoing uncertainty.
Formalising this connection (e.g.\ by considering the appropriate limits and justifying the choice of initial variance) is a matter for future work.



\section{Proofs of results}\label{sec:paper_proofs}
In this section, we provide proofs of all the results in this chapter.
We use several results from deterministic and It\^o calculus, which are stated in \Cref{app:theory} for reference.

\subsection{Preliminaries for proofs}\label{app:gauss}

There are several generic results and inequalities that we use several times throughout our proofs, which we state here for completeness.
We write \(W_t = \left(W_t^{(1)}, \hdots, W_t^{(m)}\right)^{\T}\) as the components of the canonical \(m\)-dimensional Wiener process, where each \(W_t^{(i)}\) are mutually independent 1-dimensional Wiener processes.
The flow map \(F_0^t: \R^n \to \R^n\) summarises solutions of the deterministic model \cref{eqn:ode_det} and is given by
\begin{equation}
	F_{0}^{t}(x) = x + \int_{0}^{t}{u\left(F_{0}^{\tau}(x), \tau\right)\dif \tau},
	\label{eqn:flow_map_int}
\end{equation}
for an initial condition \(x \in \R^n\).
The spatial gradient (with respect to the initial condition) of the flow map solves the equation of variations associated with \cref{eqn:ode_det}, i.e.
\begin{equation}
	\dpd{}{t}\nabla F_0^t(x) = \nabla u\left(F_0^t(x), t\right)\nabla F_0^t(x).
	\label{eqn:eqn_of_vars}
\end{equation}

For any real numbers \(x_1,\hdots,x_p \geq 0\) and \(r \geq 1\),
\begin{equation}
	\left(\sum_{i=1}^p{x_i}\right)^r \leq p^{r-1}\sum_{i=1}^p{x_i^r}.
	\label{eqn:trinomial}
\end{equation}
This results from an application of the finite form of Jensen's inequality.
An implication of the equivalence of the \(L_1\) and Euclidean norms and \cref{eqn:trinomial} is that for any \(z \in \R^n\) and \(r \geq 1\),
\begin{equation}
	\norm{z}^r \leq \left(\sum_{i = 1}^n{\abs{z_i}}\right)^r \leq n^{r-1}\sum_{i=1}^n{\abs{z_i}^r},
	\label{eqn:norm_trinomial}
\end{equation}
where \(z_i\) denotes the \(i\)th component of \(z\).
If each component \(z_i\) of a vector \(z\) is bounded by a constant \(K\), then
\begin{equation}\label{eqn:bound_vector}
	\norm{z} \leq \sqrt{n} K.
\end{equation}
Similarly, if \(f: \R \to \R^n\) is a vector-valued function such that each component of \(f\) is integrable over an interval \([0,t]\), then for all \(r \geq 1\),
\begin{equation}
	\norm{\int_0^t{f\left(\tau\right)\dif\tau}}^r \leq t^{r-1}\int_0^t{\norm{f\left(\tau\right)}^r\dif\tau}.
	\label{eqn:convex_integral}
\end{equation}
This inequality results from an application of H\"{o}lder's inequality.


%%%%%%%%%%%%%%%%%%%%%%%%%%%%%%%%%%%%%%%%%%%%%%%%%%%%%%%%%%%
\subsection{Proof of \Cref{thm:main}}\label{app:main_thm_proof}
To prove the main result, we first require a lemma establishing a bound on the time integral of the expectation of the distance between the SDE solution and the reference deterministic trajectory.

\begin{lemma}\label{lem:z_int_bound}
	Let \(q \geq 1\) be such that \(\delta_q < \infty\), then for all \(\epsilon > 0\) and \(\tau \in [0,T]\)
	\begin{equation*}
		\avg{\int_0^t{\norm{y_\tau^{(\epsilon)} - F_0^t\!\left(x_0\right)}^q\dif\tau}} \leq H_1\!\left(q,t, K_{\nabla u}, K_{\sigma}\right)\epsilon^q + H_2\!\left(q,t, K_{\nabla u}\right)\delta_q^q,
	\end{equation*}
	where
	\begin{align*}
		H_1\!\left(q,t, K_{\nabla u}, K_{\sigma}\right) & \coloneqq 3^{q-1} n^{3q/2} K_{\sigma}^{q/2} G_{q/2} t^{q/2 + 1}\exp\left(3^{q-1} K_{\nabla u}^q t^q\right), \\
		H_2\!\left(q,t, K_{\nabla u}\right)             & \coloneqq 3^{q-1} t \exp\left(3^{q-1} K_{\nabla u}^q t^q\right).
	\end{align*}
\end{lemma}

\begin{proof}
	Consider the integral form of \cref{eqn:sde_y},
	\[
		y_t^{(\epsilon)} = x + \int_0^t{u\left(y_\tau^{(\epsilon)}, \tau\right)\dif\tau} + \epsilon\int_0^t{\sigma\left(y_\tau^{(\epsilon)}, \tau\right)\dif W_\tau}.
	\]
	Using \cref{eqn:flow_map_int},
	\[
		y_t^{(\epsilon)} - F_0^t\!\left(x_0\right) = x - x_0 + \int_0^{t}{\left(u\left(y_\tau^{(\epsilon)}, \tau\right) - u\left(F_0^{\tau}\!\left(x_0\right), \tau\right)\right)\dif\tau} + \epsilon\int_0^t{\sigma\left(y_\tau^{(\epsilon)}, \tau\right)\dif W_\tau},
	\]
	and so
	\begin{equation}
		\avg{\norm{y_t^{(\epsilon)} - F_0^t\!\left(x_0\right)}^q} \leq \begin{multlined}[t]
			3^{q-1}\avg{\norm{x - x_0}^q} \\
			+ 3^{q-1}t^{q-1}\avg{\int_0^{t}{\norm{u\!\left(y_\tau^{(\epsilon)}, \tau\right) - u\left(F_0^{\tau}\!\left(x_0\right), \tau\right)}^q\dif\tau}} \\
			+ 3^{q-1}\epsilon^q\avg{\norm{\int_0^t{\sigma\!\left(y_\tau^{(\epsilon)}, \tau\right)\dif W_\tau}}^q},
		\end{multlined}
		\label{eqn:norm_y_t_tmp}
	\end{equation}
	using \cref{eqn:trinomial} followed by \cref{eqn:convex_integral}, and taking the expectation on both sides.

	Next, we establish a bound on the It\^o integral term in \cref{eqn:norm_y_t_tmp}.
	For \(i \in \set{1,\hdots,n}\), let \(\sigma_{i\cdot}\) denote the \(i\)th row of \(\sigma\).
	Define the stochastic process
	\[
		M_\tau^{(i)} \coloneqq \sigma_{i\cdot}\!\left(y_\tau^{(\epsilon)}, \tau\right)
	\]
	for \(\tau \in [0,t]\), so that
	\[
		\left[\int_0^t{\sigma\!\left(y_\tau^{(\epsilon)}, \tau\right)\dif W_\tau}\right]_i = \int_0^t{M_\tau^{(i)}\dif W_\tau}.
	\]
	Since \(y_t^{(\epsilon)}\) is a strong solution to \cref{eqn:sde_y}, we have that (e.g.\ see Definition 6.1.1 of \citet{KallianpurSundar_2014_StochasticAnalysisDiffusion})
	\[
		\int_0^t{\norm{M_\tau^{(i)}}^2\dif\tau} \leq \int_0^t{nK_\sigma^2\dif\tau} < \infty, \quad \text{almost surely},
	\]
	so we can apply the Burkholder-Davis-Gundy inequality (see \Cref{thm:bdg}) to \(M_\tau\), which asserts that there exists a constant \(G_{q/2} > 0\) depending only on \(q\) such that
	\begin{align*}
		\avg{\abs{\int_0^t{M_\tau^{(i)}\dif W_\tau}}^{q}} & \leq G_{q/2} \avg{\left(\int_{0}^t{\norm{\sigma_{i\cdot}\!\left(y_\tau^{(\epsilon)}, \tau\right)}^2\dif\tau}\right)^{q/2}} \\
		                                                  & \leq G_{q/2} n^p K_{\sigma}^{q/2} t^{q/2},
	\end{align*}
	where the second inequality uses \ref{hyp:sigma_bounds}.
	Then,
	\begin{equation}
		\avg{\norm{\int_0^t{\sigma\!\left(y_\tau^{(\epsilon)}, \tau\right)\dif W_\tau}}^{q}} \leq n^{3q/2} K_\sigma^{q/2} G_{q/2} t^{q/2},
		\label{eqn:z_sigma_bound}
	\end{equation}
	using \cref{eqn:norm_trinomial}.

	Applying the bound \cref{eqn:z_sigma_bound} to \cref{eqn:norm_y_t_tmp}, we have
	\begin{equation}\label{eqn:y_F_diff_inter}
		\avg{\norm{y_t^{(\epsilon)}\!-\!F_0^t\!\left(x_0\right)}^q} \leq \begin{multlined}[t]
			3^{q-1}\delta_q^q + 3^{q-1}\epsilon^q n^{3q/2}K_{\sigma}^{q/2} G_{q/2} t^{q/2} \\
			+ 3^{q-1}t^{q-1}\avg{\int_0^{t}{\norm{u\!\left(y_\tau^{(\epsilon)}, \tau\right) - u\!\left(F_0^{\tau}\!\left(x_0\right), \tau\right)}}^q\dif\tau}.
		\end{multlined}
	\end{equation}
	We note that \(\avg{\norm{y_t^{(\epsilon)} - F_0^t\!\left(x\right)}^q} < \infty\) from \ref{hyp:u_bounds}, so by Tonelli's theorem (e.g.\ \citet[Thm. 2.3.9]{Bremaud_2020_ProbabilityTheoryStochastic}),
	\begin{equation*}
		\avg{\int_0^{t}{\norm{y_\tau^{(\epsilon)} - F_0^\tau\!\left(x_0\right)}^q\dif\tau}} = \int_0^{t}{\avg{\norm{y_\tau^{(\epsilon)} - F_0^\tau\!\left(x_0\right)}^q}\dif\tau}.
		\label{eqn:y_tonelli}
	\end{equation*}
	Now, using the Lipschitz continuity of \(u \) from \ref{hyp:sigma_deriv_bound} on \cref{eqn:y_F_diff_inter} and interchanging the expectation and integral,
	\[
		\avg{\norm{y_t^{(\epsilon)} - F_0^t\!\left(x_0\right)}^q} \leq \begin{multlined}[t]
			3^{q-1}\delta_q^q + 3^{q-1} K_{\nabla u}^q t^{q-1} \int_0^{t}{\avg{\norm{y_\tau^{(\epsilon)} - F_0^{\tau}\!\left(x_0\right)}^q}\dif\tau} \\
			+ 3^{q-1}\epsilon^q n^{3q/2}K_{\sigma}^{q/2} G_{q/2} t^{q/2}.
		\end{multlined}
	\]
	Applying Gr\"{o}nwall's inequality and using the monotonicity of the resulting bound in \(t\), we have that for any \(\tau \in [0,t]\),
	\[
		\avg{\norm{y_\tau^{(\epsilon)} - F_0^\tau\!\left(x_0\right)}^q}  \leq \begin{multlined}[t]
			3^{q-1}\epsilon^q n^{3q/2}S^{q/2} G_{q/2} t^{q/2}\exp\left(3^{q-1} K_{\nabla u}^q t^q\right) \\
			+ 3^{q-1}\exp\!\left(3^{q-1} K_{\nabla u}^q t^q\right)\delta_q^q .
		\end{multlined}
	\]
	Integrating both sides with respect to time
	and again using Tonelli's theorem, we have
	\begin{align*}
		\avg{\int_0^t{\norm{y_\tau^{(\epsilon)} - F_0^\tau\!\left(x_0\right)}^q\dif\tau}} \leq \begin{multlined}[t]
			                                                                                       3^{q-1}n^{3q/2}S^{q/2} G_{q/2} t^{q/2 + 1}\exp\!\left(3^{q-1} K_{\nabla u}^q t^q\right)\epsilon^q  \\
			                                                                                       + 3^{q-1}t\exp\left(3^{q-1} K_{\nabla u}^q t^q\right)\delta_q^q ,
		                                                                                       \end{multlined}
	\end{align*}
	as desired.
\end{proof}

With these bounds established, we can now prove \Cref{thm:main}.
Subtracting the integral forms of \cref{eqn:linear_sde_inform} and \cref{eqn:sde_y} gives
\begin{align*}
	y_t^{(\epsilon)} - l_t^{(\epsilon)} & = \begin{multlined}[t]
		                                        \int_0^t{\left[u\!\left(y_\tau^{(\epsilon)}, \tau\right) - u\!\left(F_0^\tau\!\left(x_0\right), \tau\right) - \nabla u\!\left(F_0^\tau\!\left(x_0\right)\right)\left(l_\tau^{(\epsilon)} - F_0^\tau\!\left(x_0\right)\right)\right]\dif\tau} \\
		                                        + \int_0^t{\left[\epsilon\sigma\!\left(y_\tau^{(\epsilon)}, \tau\right) - \epsilon\sigma\!\left(F_0^{\tau}\!\left(x_0\right), \tau\right)\right]\dif W_\tau}
	                                        \end{multlined}              \\
	                                    & = \begin{multlined}[t]
		                                        \int_0^t{\left[u\!\left(y_\tau^{(\epsilon)}, \tau\right) - \left(u\!\left(F_0^\tau\!\left(x_0\right), \tau\right) + \nabla u\!\left(F_0^\tau\!\left(x_0\right)\right)\left(y_\tau^{(\epsilon)} - F_0^\tau\!\left(x_0\right)\right)\right)\right]\dif\tau} \\
		                                        + \int_0^t{\nabla u\!\left(F_0^\tau\!\left(x_0\right), \tau\right)\left[y_\tau^{(\epsilon)} - l_\tau^{(\epsilon)}\right]\dif \tau} \\
		                                        + \epsilon\int_0^t{ \left[\sigma\!\left(y_\tau^{(\epsilon)}, \tau\right) - \sigma\!\left(F_0^{\tau}\!\left(x_0\right), \tau\right)\right]\dif W_\tau}
	                                        \end{multlined} \\
	                                    & = A(t) + B(t) + \epsilon C(t),
\end{align*}
where
\begin{align*}
	A(t) & \coloneqq \int_0^t{\left[u\!\left(y_\tau^{(\epsilon)}, \tau\right) - \left(u\!\left(F_0^\tau\!\left(x_0\right), \tau\right) + \nabla u\!\left(F_0^\tau\!\left(x_0\right)\right)\left(y_\tau^{(\epsilon)} - F_0^\tau\!\left(x_0\right)\right)\right)\right]\dif\tau} \\
	B(t) & \coloneqq \int_0^t{\nabla u\!\left(F_0^\tau\!\left(x_0\right), \tau\right)\left[y_\tau^{(\epsilon)} - l_\tau^{(\epsilon)}\right]\dif \tau}                                                                                                                          \\
	C(t) & \coloneqq \int_0^t{\left[\sigma\!\left(y_\tau^{(\epsilon)}, \tau\right) - \sigma\!\left(F_0^{\tau}\!\left(x_0\right), \tau\right)\right]\dif W_\tau}.
\end{align*}
Then, using \cref{eqn:trinomial} and taking expectation,
\begin{equation}
	\avg{\norm{y_t^{(\epsilon)} - l_t^{(\epsilon)}}^r} \leq 3^{r-1}\left(\avg{\norm{A(t)}^r} + \avg{\norm{B(t)}^r} + \epsilon^r\avg{\norm{C(t)}^r}\right) .
	\label{eqn:diff_decomp}
\end{equation}
First consider \(A(t)\), for which the integrand is the expected difference between the drift \(u\) evaluated along SDE solution and the first-order Taylor expansion of \(u\) about the reference deterministic trajectory.
Since for any \(t \in [0,T]\), \(u\left(\cdot, t\right)\) is twice continuously differentiable under \ref{hyp:coef_cont}, for each \(i = 1,\hdots,n\) there exists by Taylor's theorem (e.g.\ see \citet[Cor. A9.3.]{HubbardHubbard_2009_VectorCalculusLinear}) a function \(R_i: \R^n \times [0,T] \to \R\) such that
\begin{equation}
	u_i\!\left(z, \tau\right) = u_i\!\left(F_0^\tau\!\left(x_0\right), \tau\right) + \left[\nabla u_i\left(F_0^\tau\!\left(x_0\right), \tau\right)\right]\left(z - F_0^\tau\!\left(x_0\right)\right) + R_i\!\left(z, \tau\right)
	\label{eqn:taylor_expan}
\end{equation}
for any \(z \in \R^n\), where \(u_i\) denotes the \(i\)th component of \(u\).
The function \(R_i\) satisfies
\begin{equation}
	\abs{R_i(z, \tau)} \leq \frac12\norm{\nabla\nabla u_i\left(F_0^\tau(x), t\right)}\norm{z - F_0^\tau\!\left(x_0\right)}^2 \leq \frac{K_{\nabla\nabla u}}{2}\norm{z - F_0^\tau\!\left(x_0\right)}^2.
	\label{eqn:rem_ineq}
\end{equation}
Let \(R\!\left(z, \tau\right) \coloneqq \left( R_1\!\left(z, \tau\right), \hdots, R_n\!\left(z, \tau\right)\right)^{\T}\), then
\[
	A(t) = \int_0^t{R\left(y_{t}^{(\epsilon)}, \tau\right)\dif\tau},
\]
and since each component of \(R\) is bounded as in \cref{eqn:rem_ineq}, using \cref{eqn:bound_vector}
\[
	\norm{R\!\left(y_t^{(\epsilon)}, \tau\right)} \leq \frac{\sqrt{n} K_{\nabla\nabla u}}{2}\norm{y_t^{(\epsilon)} - F_0^\tau\!\left(x_0\right)}^2.
\]
Taking the norm and expectation then gives
\begin{align*}
	\avg{\norm{A(t)}^r} & = \avg{\norm{\int_0^t{R\!\left(y_t^{(\epsilon)}, \tau\right)\dif\tau}}^r}                                                                                                                 \\
	                    & \leq t^{r-1}\avg{\int_0^t{\norm{R\!\left(y_\tau^{(\epsilon)}, \tau\right)}^r\dif\tau}}                                                                                                    \\
	                    & \leq \frac{t^{r-1}n^{r/2}K_{\nabla\nabla u}^r}{2^r}\avg{\int_0^t{\norm{y_\tau^{(\epsilon)} - F_0^\tau\!\left(x_0\right)}^{2r}\dif\tau}}                                                   \\
	                    & \leq \begin{multlined}[t]
		                           \frac{t^{r-1} n^{r/2} K^r_{\nabla\nabla u}H_1\!\left(2r,t, K_{\nabla u}, K_{\sigma}\right)}{2^r}\epsilon^{2r} \\
		                           + \frac{t^{r-1} n^{r/2} K_{\nabla\nabla u}^r H_2\!\left(2r,t, K_{\nabla u}\right)}{2^{r}}\delta_{2r}^{2r},
	                           \end{multlined}\numberthis\label{eqn:A_ineq}
\end{align*}
where the first inequality uses \cref{eqn:convex_integral}, and \(H_1\) and \(H_2\) are obtained from \Cref{lem:z_int_bound}.

Next, consider \(B(t)\), for which
\begin{equation}\label{eqn:B_ineq}
	\avg{\norm{B(t)}^r} \leq \int_0^t{t^{r-1}K_{\nabla u}^r\avg{\norm{y_\tau^{(\epsilon)} - l_t^{(\epsilon)}}^r}\dif\tau}.
\end{equation}
using \cref{eqn:convex_integral} and then \ref{hyp:u_bounds}, and interchanging the expectation and the integral uses the fact that that \(\avg{\norm{y_\tau^{(\epsilon)}}} < \infty\) and \(\avg{\norm{l_\tau^{(\epsilon)}}} < \infty\).

Finally, consider \(C(t)\).
For each \(i \in \set{1,\hdots, n}\), define the stochastic process
\[
	N_\tau^{(i)} \coloneqq \sigma_{i\cdot}\left(y_\tau^{(\epsilon)},\tau\right) - \sigma_{i\cdot} \left(F_0^\tau\!\left(x_0\right), \tau\right).
\]
Then, the \(i\)th component of \(C(t)\) is
\[
	\left[C(t)\right]_i = \int_0^t{N_\tau^{(i)}\dif W_\tau}.
\]
From \ref{hyp:sigma_bounds} and using \cref{eqn:bound_vector},
\[
	\int_0^t{\norm{N_\tau^{(i)}}^2\dif\tau} \leq \int_0^t{4nK_\sigma^2\dif\tau} < \infty,
\]
so we can apply the Burkholder-Davis-Gundy inequality on \(N_\tau^{(i)}\) to write
\begin{align*}
	\avg{\abs{\left[C(t)\right]_i}^{r}} & \leq G_{r/2}\avg{\left(\int_{0}^t{\norm{\sigma_{i\cdot}\left(y_\tau^{(\epsilon)}, \tau\right) - \sigma_{i\cdot} \left(F_0^\tau\!\left(x_0\right), \tau\right)}^2\dif\tau}\right)^{r/2}} \\
	                                    & \leq G_{r/2}\avg{\left(\int_0^{t}{ K_{\nabla\sigma}^2\norm{y_\tau^{(\epsilon)} - F_0^\tau\!\left(x_0\right)}^2\dif\tau}\right)^{r/2}}                                                   \\
	                                    & \leq G_{r/2} K_{\nabla \sigma}^r t^{r/2 - 1}\avg{\int_0^t{\norm{y_\tau^{(\epsilon)} - F_0^\tau\!\left(x_0\right)}^r\dif\tau}}                                                           \\
	                                    & \leq \begin{multlined}[t]
		                                           G_{r/2} K_{\nabla\sigma}^r t^{r/2 - 1} H_1\!\left(r,t, K_{\nabla u}, K_{\sigma}\right)\epsilon^r \\
		                                           + G_{r/2} K_{\nabla\sigma}^r t^{r/2 - 1} H_2\!\left(r,t, K_{\nabla u}\right)\delta_r^r,
	                                           \end{multlined} \numberthis\label{eqn:N_comp_ineq}
\end{align*}
where the second inequality uses the Lipschitz condition on \(\sigma\) in \ref{hyp:sigma_deriv_bound}, the third inequality uses \cref{eqn:convex_integral}, and the fourth inequality uses \Cref{lem:z_int_bound} with \(q = r\).
Then, we have
\begin{equation}\label{eqn:C_ineq}
	\avg{\norm{C(t)}^r} \leq \begin{multlined}[t]
		n^{r}G_{r/2} K_{\nabla\sigma}^r t^{r/2 - 1} H_1\!\left(r,t, K_{\nabla u}, K_{\sigma}\right) \epsilon^r \\
		+ n^r G_{r/2} K_{\nabla\sigma}^r t^{r/2 - 1} H_2\!\left(r,t, K_{\nabla u}\right)\delta_r^r,
	\end{multlined}
\end{equation}
using \cref{eqn:norm_trinomial}, and then \cref{eqn:N_comp_ineq}.


Combining \cref{eqn:A_ineq}, \cref{eqn:B_ineq} and \cref{eqn:C_ineq} into \cref{eqn:diff_decomp}, we have
\[
	\avg{\norm{y_t^{(\epsilon)} - l_t^{(\epsilon)}}^r} \leq \begin{multlined}[t]
		\frac{3^{r-1} t^{r-1}n^{r/2}K_{\nabla\nabla u}^r H_1\!\left(2r,t, K_{\nabla u}, K_{\sigma}\right)}{2^r}\epsilon^{2r} \\
		+ \frac{3^{r-1} t^{r-1} n^{r/2} K_{\nabla\nabla u}^r H_2\!\left(2r,t, K_{\nabla u}\right)}{2^r}\delta_{2r}^{2r} \\
		+ \int_0^t{3^{r-1}t^{r-1}K_{\nabla u}^r\avg{\norm{y_t^{(\epsilon)} - l_t^{(\epsilon)}}^r}\dif\tau} \\
		+ 3^{r-1}G_{r/2} K_{\nabla\sigma}^r t^{r/2 - 1} H_1\!\left(r,t, K_{\nabla u}, K_{\sigma}\right)\epsilon^{2r} \\
		+ 3^{r-1} G_{r/2} K_{\nabla\sigma}^r t^{r/2 - 1} H_2\!\left(r,t, K_{\nabla u}\right)\epsilon^r\delta_r^r.
	\end{multlined}
\]
Applying Gr\"{o}nwall's inequality, noting that \(H_1\) and \(H_2\) are non-decreasing in \(t\), we have
\[
	\avg{\norm{y_t^{(\epsilon)} - l_t^{(\epsilon)}}^r} \leq \begin{multlined}[t]
		\frac{3^{r-1} t^{r-1}n^{r/2}K_{\nabla\nabla u}^r H_1\!\left(2r,t, K_{\nabla u}, K_{\sigma}\right)}{2^r}\exp\left(3^{r-1}t^r K_{\nabla u}^r\right)\epsilon^{2r} \\
		+ \frac{3^{r-1} t^{r-1} n^{r/2} K_{\nabla\nabla u}^r H_2\!\left(2r,t, K_{\nabla u}\right)}{2^r}\exp\left(3^{r-1}t^r K_{\nabla u}^r\right)\delta_{2r}^{2r} \\
		+ 3^{r-1}G_{r/2} K_{\nabla\sigma}^r t^{r/2 - 1}\exp\left(3^{r-1}t^r K_{\nabla u}^r\right) H_1\!\left(r,t, K_{\nabla u}, K_{\sigma}\right)\epsilon^{2r} \\
		+ 3^{r-1} G_{r/2} K_{\nabla\sigma}^r t^{r/2 - 1}\exp\left(3^{r-1}t^r K_{\nabla u}^r\right) H_2\!\left(r,t, K_{\nabla u}\right)\epsilon^r\delta_r^r.
	\end{multlined}
\]
Set
\begin{subequations}\label{eqn:bound_defns}
	\begin{align}
		D_1\!\left(r,t, K_{\nabla u}, K_\sigma\right) & \coloneqq 3^{r-1}\exp\left(3^{r-1} t^r K_{\nabla u}^r\right)K_{M}\!\left(r, t, K_{\nabla u}, K_\sigma\right)        \\
		D_2\!\left(r,t, K_{\nabla u}\right)           & \coloneqq 3^{r-1}t^{r-1} n^{r/2}H_{2}\!\left(2r,t, K_{\nabla u}\right)\exp\left(3^{r-1}t^r K_{\nabla u}^r\right)    \\
		D_3\!\left(r,t, K_{\nabla u}\right)           & \coloneqq 3^{r-1}G_{r/2} t^{r/2 - 1} H_2\!\left(r,t, K_{\nabla u}\right)\exp\left(3^{r-1}t^r K_{\nabla u}^r\right),
	\end{align}
\end{subequations}
where
\[
	K_{M}\!\left(r, t, K_{\nabla u}, K_\sigma\right) \coloneqq \max\set{\frac{t^{r-1}n^{r/2}H_{1}\!\left(2r,t, K_{\nabla u}, K_{\sigma}\right)}{2^r},\, G_{r/2} t^{r/2 - 1} H_1\!\left(r,t, K_{\nabla u}, K_{\sigma}\right)},
\]
then we have shown the desired result.


%%%%%%%%%%%%%%%%%%%%%%%%%%%%%%%%%%%%%%%%%%%%%%%%%%%%
\subsection{Proof of \Cref{thm:limit_sol}}\label{app:limit_sol_proof}

Next, we show that the strong solution to the linearised SDE \cref{eqn:linear_sde_inform} can be written as the independent sum \cref{eqn:linear_sol}.
Let
\[
	M_t = h\!\left(l_t^{(\epsilon)}, t\right) \coloneqq \frac{1}{\epsilon}\left[\nabla F_0^t\!\left(x_0\right)\right]^{-1}\left(l_t^{(\epsilon)} - F_0^t\!\left(x_0\right)\right),
\]
where \(l_t^{(\epsilon)}\) is the strong solution to \cref{eqn:linear_sde_inform}.
Then, the required derivatives for applying It\^o's Lemma (see \Cref{thm:ito_lemma}) are
\begin{align*}
	M_0                                              & = h\!\left(l_0^{(\epsilon)}, 0\right) = \frac{1}{\epsilon}\left(x - x_0\right)                                                                                                                                       \\
	\dpd{h}{t}                                       & = \begin{multlined}[t]
		                                                     -\frac{1}{\epsilon}\left[\nabla F_0^t\!\left(x_0\right)\right]^{-1} \dpd{\nabla F_0^t\!\left(x_0\right)}{t}\left[\nabla F_0^t\!\left(x_0\right)\right]^{-1}\left(l_t^{(\epsilon)} - F_0^t\!\left(x_0\right)\right) \\
		                                                     - \frac{1}{\epsilon}\left[\nabla F_0^t\!\left(x_0\right)\right]^{-1}u\!\left(F_0^t\!\left(x_0\right), t\right)
	                                                     \end{multlined} \\
	\nabla h\!\left(l_t^{(\epsilon)}, t\right)       & = \frac{1}{\epsilon}\left[\nabla F_0^t\!\left(x_0\right)\right]^{-1}                                                                                                                                                 \\
	\nabla\nabla h\!\left(l_t^{(\epsilon)}, t\right) & = O,
\end{align*}
where in computing the \(t\) derivative we have used the fact that \(F_0^t\!\left(x_0\right)\) solves the deterministic ODE \cref{eqn:ode_det}.
Thus,
\begin{align*}
	M_t & = \begin{multlined}[t]
		        \frac{1}{\epsilon}\left(x - x_0\right) + \frac{1}{\epsilon}\int_0^t\left(-\left[\nabla F_0^\tau\!\left(x_0\right)\right]^{-1} \dpd{\nabla F_0^\tau\!\left(x_0\right)}{\tau}\left[\nabla F_0^\tau\!\left(x_0\right)\right]^{-1}\left(l_\tau^{(\epsilon)} - F_0^\tau\!\left(x_0\right) \right)\right. \\
		        \left. - \left[\nabla F_0^\tau\!\left(x_0\right)\right]^{-1}u\!\left(F_0^\tau\!\left(x_0\right), \tau\right)\right. \\
		        \left. + \left[\nabla F_0^\tau\!\left(x_0\right)\right]^{-1}\left[ u\!\left(F_0^\tau\!\left(x_0\right), \tau\right) + \nabla u\!\left(F_0^\tau\!\left(x_0\right), \tau\right) \left(l_\tau^{(\epsilon)} - F_0^\tau\!\left(x_0\right)\right)\right] \right)\dif\tau \\
		        + \int_0^t{\left[\nabla F_0^\tau\!\left(x_0\right)\right]^{-1}\sigma\!\left(F_0^\tau\!\left(x_0\right), \tau\right)\dif W_\tau}
	        \end{multlined} \\
	    & = \begin{multlined}[t]
		        \frac{1}{\epsilon}\left(x - x_0\right) + \frac{1}{\epsilon}\int_0^t\left(-\left[\nabla F_0^\tau\!\left(x_0\right)\right]^{-1} \nabla u\!\left(F_0^\tau\!\left(x_0\right), \tau\right)\left(l_\tau^{(\epsilon)} - F_0^\tau\!\left(x_0\right) \right)\right. \\
		        \left. + \left[\nabla F_0^\tau\!\left(x_0\right)\right]^{-1}\nabla u\!\left(F_0^\tau\!\left(x_0\right), \tau\right) \left(l_\tau^{(\epsilon)} - F_0^\tau\!\left(x_0\right)\right) \right)\dif\tau \\
		        + \int_0^t{\left[\nabla F_0^\tau\!\left(x_0\right)\right]^{-1}\sigma\!\left(F_0^\tau\!\left(x_0\right), \tau\right)\dif W_\tau}
	        \end{multlined}              \\
	    & = \frac{1}{\epsilon}\left(x - x_0\right) + \int_0^t{\left[\nabla F_0^\tau\!\left(x_0\right)\right]^{-1}\sigma\!\left(F_0^\tau\!\left(x_0\right), \tau\right)\dif W_\tau},
\end{align*}
where we reach the second line by using the equation of variations \cref{eqn:eqn_of_vars} satisfied by \(\nabla F_0^\tau\!\left(x_0\right)\).
It follows that the expression \cref{eqn:linear_sol}, i.e.
\[
	l_t^{(\epsilon)} = \nabla F_0^t\!\left(x_0\right)\left(x - x_0\right) +  F_0^t\!\left(x_0\right) + \epsilon\int_0^t{\nabla F_0^t\!\left(x_0\right)\left[\nabla F_0^\tau\!\left(x_0\right)\right]^{-1} \sigma\!\left(F_0^\tau\!\left(x_0\right), \tau\right)\dif \tau},
\]
is a strong solution to \cref{eqn:linear_sde_inform}.

Finally, under \ref{hyp:init_indep}, \(x\) is independent of the Wiener process \(W_t\), and since independence is preserved under limits and linear transformations, it follows that the two random terms in \cref{eqn:linear_sol} are independent.


%%%%%%%%%%%%%%%%%%%%%%%%%%%%%%%%%%%%%%%%%%%%%%%%%%%%
\subsection{Proof of \Cref{cor:limit_moments}}\label{app:limit_moments_proof}
We first establish that the It\^o integral of a matrix-valued deterministic function with respect to a multidimensional Wiener process is a multidimensional Gaussian process.
This is a well-known result in the scalar case, and the extension to our case is straightforward.
\begin{lemma}\label{lem:det_gauss}
	Let \(a,b \in \R\) and let \(g: [a,b] \to \R^{n\times n}\) be a matrix-valued deterministic function such that each element of \(g\) is It\^o-integrable.
	Consider the It\^o integral
	\[
		\mathcal{I}[g] \coloneqq \int_{a}^b{g(t)\dif W_t},
	\]
	Then, the integral \(\mathcal{I}[g]\) is a \(n\)-dimensional multivariate Gaussian random variable.
\end{lemma}
\begin{proof}
	For \(i,j \in \set{1,\hdots,n}\), let \(g_{ij}: [a,b] \to \R\) be the \((i,j)\)th element of \(g\).
	Then, let
	\[
		\mathcal{I}[g_{ij}] \coloneqq \int_a^b{g_{ij}(t)\dif W_t^{(i)}},
	\]
	so that the \(i\)th element of \(\mathcal{I}[g]\) is
	\[
		\mathcal{I}[g]_i = \sum_{j = 1}^n{\mathcal{I}\left[g_{ij}\right]}.
	\]
	Each \(\mathcal{I}[g_{ij}]\) is an It\^o integral of a deterministic, scalar-valued function with respect to a one-dimensional Brownian motion, which is well-known to be a Gaussian process (e.g.\ see Lemma 4.3. of \citet{Applebaum_2004_LevyProcessesStochastic}).
	Moreover, each element of \(\mathcal{I}[g]\) is the sum of independent Gaussian random variables and is therefore itself Gaussian.
	Hence, \(\mathcal{I}[g]\) follows a multivariate Gaussian distribution.
\end{proof}

Now, we move onto showing \Cref{cor:limit_moments}.
Consider the It\^o integral
\[
	\mathcal{I}[L] = \int_0^t{L\!\left(x_0, \tau\right)\dif W_\tau}.
\]
For any fixed \(t \in [0,T]\), the integrand is a deterministic, matrix-valued function, and is therefore follows a \(n\)-dimensional Gaussian distribution.
Moreover \citep{KallianpurSundar_2014_StochasticAnalysisDiffusion},
\[
	\avg{\mathcal{I}[L]} = 0,
\]
and
\[
	\var{\mathcal{I}[L]} = \avg{\left(\int_0^t{L\!\left(x_0, \tau\right)\dif W_\tau}\right)\left(\int_0^t{L\!\left(x_0, \tau\right)\dif W_\tau}\right)^{\T}}.
\]
Let \(L_{ij}\) denote the \((i,j)\)th element of \(L\), then the \((i,j)\)th element of the variance is
\begin{align*}
	\left[\var{\mathcal{I}[L]}\right]_{ij} % & = \avg{\left(\sum_{k=1}^m\int_0^t{L_{ik}\!\left(x_0, \tau\right)\dif W_{\tau}^{(k)}}\right)\left(\sum_{l=1}^m\int_0^t{L_{jl}\!\left(x_0, \tau\right)\dif W_{\tau}^{(l)}}\right)}   \\
	 & = \sum_{k=1}^{m}\sum_{l=1}^m\avg{\left(\int_0^t{L_{ik}\!\left(x_0, \tau\right)\dif W_{\tau}^{(k)}}\right)\left(\int_0^t{L_{jl}\!\left(x_0, \tau\right)\dif W_{\tau}^{(l)}}\right)} \\
	 & = \begin{multlined}[t]
		     \sum_{k=1}^{m}\avg{\left(\int_0^t{L_{ik}\!\left(x_0, \tau\right)\dif W_{\tau}^{(k)}}\right)\!\left(\int_0^t{L_{jk}\!\left(x_0, \tau\right)\dif W_{\tau}^{(k)}}\right)} \\
		     \!\!+ \sum_{k=1}^{m}\sum_{\substack{l=1 \\ l \neq k}}^m\avg{\left(\int_0^t{L_{ik}\!\left(x_0, \tau\right)\dif W_{\tau}^{(k)}}\right)}\!\avg{\left(\int_0^t{L_{jl}\!\left(x_0, \tau\right)\dif W_{\tau}^{(l)}}\right)}
	     \end{multlined}                                                                  \\
	 & = \sum_{k=1}^m{\int_0^t{L_{ik}\!\left(x_0, \tau\right)L_{jk}\!\left(x_0, \tau\right)\dif\tau}},
\end{align*}
where the second equality  the fact that \(W_t^{(k)}\) is independent of \(W_t^{(l)}\) for \(k \neq l\) and the third equality uses It\^o's isometry \citep{KallianpurSundar_2014_StochasticAnalysisDiffusion}.
Hence, we have that
\[
	\int_0^t{L\!\left(x_0, \tau\right)\dif\tau} \isGauss{0, \int_0^t{L\!\left(x_0, \tau\right)L\!\left(x_0, \tau\right)^{\T}}\dif\tau},
\]
completing the proof of \Cref{thm:limit_sol}.
Next, we show that the mean and covariance of \(l_t^{(\epsilon)}\) are given explicitly by \cref{eqn:mean_expl_eqn} and \cref{eqn:pi_expl_eqn} respectively.
It follows immediately from \cref{eqn:linear_sol} that the mean of \(l_t^{(\epsilon)}\) is
\begin{align*}
	\avg{l_t^{(\epsilon)}} & = \avg{\nabla F_0^t\!\left(x_0\right) \left(x - x_0\right)} + F_0^t\!\left(x_0\right) + \epsilon \nabla F_0^t\!\left(x_0\right)\avg{\int_0^t{L\left(x_0,\tau\right)\dif\tau}} \\
	                       & = \nabla F_0^t(x) \left(\avg{x} - x_0\right) + F_0^t\!\left(x_0\right),
\end{align*}
thus showing \cref{eqn:mean_expl_eqn}.

Since the two summands in \cref{eqn:linear_sol} are independent, the variance of \(l_t^{(\epsilon)}\) is
\begin{align*}
	\var{l_t^{(\epsilon)}} & = \var{\nabla F_0^t\!\left(x_0\right) \left(x - x_0\right)} + \var{\epsilon \nabla F_0^t\!\left(x_0\right)\int_0^t{L\!\left(x_0,\tau\right)\dif W_\tau}}                                      \\
	                       & = \nabla F_0^t\!\left(x_0\right) \left(\var{x} + \epsilon^2\int_0^t{L\!\left(x_0, \tau\right)L\!\left(x_0, \tau\right)^{\T} \dif\tau}\right) \left[\nabla F_0^t\!\left(x_0\right)\right]^{\T}
\end{align*}
where the variance of the It\^o integral was established in \Cref{app:limit_sol_proof}.

Finally, we show \Cref{rem:cov_ode}, i.e.\ that \(\var{l_t^{(\epsilon)}}\) is the solution to the matrix differential equation \cref{eqn:pi_ode}.
Directly differentiating the expression \cref{eqn:pi_expl_eqn}
\begin{align*}
	\dod{\var{l_t^{(\epsilon)}}}{t} & = \begin{multlined}[t]
		                                    \dpd{\nabla F_0^t\!\left(x_0\right)}{t}\left(\var{x} + \epsilon^2 \int_0^t{L\!\left(x_0, \tau\right)L\!\left(x_0, \tau\right)^{\T}\dif\tau} \right)\left[\nabla F_0^t\!\left(x_0\right)\right]^{\T} \\
		                                    + \nabla F_0^t\!\left(x_0\right)\left(\var{x} + \epsilon^2 \int_0^t{L\!\left(x_0, \tau\right)L\!\left(x_0, \tau\right)^{\T}\dif\tau} \right)\left[\dpd{\nabla F_0^t\!\left(x_0\right)}{t}\right]^{\T} \\
		                                    + \epsilon^2\nabla F_0^t\!\left(x_0\right)L\!\left(x_0, t\right) L\!\left(x_0, t\right)^{\T}\left[\nabla F_0^t\!\left(x_0\right)\right]^{\T}
	                                    \end{multlined}                                                                   \\
	                                & = \begin{multlined}[t]
		                                    \nabla u\!\left(F_0^t\!\left(x_0\right), t\right)\nabla F_0^t\!\left(x_0\right)\left(\var{x} + \epsilon^2\!\int_0^t{L\!\left(x_0, \tau\right)L\!\left(x_0, \tau\right)^{\T}\dif\tau} \right)\left[\nabla F_0^t\!\left(x_0\right)\right]^{\T} \\
		                                    +\nabla F_0^t\!\left(x_0\right)\!\left(\var{x} + \epsilon^2\!\int_0^t{L\!\left(x_0, \tau\right)L\!\left(x_0, \tau\right)^{\T}\dif\tau} \right)\!\left[\nabla F_0^t\!\left(x_0\right)\right]^{\T}\!\left[\nabla u\!\left(F_0^t\!\left(x_0\right), t\right)\right]^{\T} \\
		                                    + \epsilon^2\sigma\!\left(F_0^t\!\left(x_0\right), t\right)\!\sigma\!\left(F_0^t\!\left(x_0\right), t\right)^{\T}
	                                    \end{multlined} \\
	                                & = \begin{multlined}[t]
		                                    \nabla u\!\left(F_0^t\!\left(x_0\right), t\right) \var{l_t^{(\epsilon)}} + \var{l_t^{(\epsilon)}}\left[\nabla u\!\left(F_0^t\!\left(x_0\right), t\right)\right]^{\T} \\
		                                    + \epsilon^2\sigma\!\left(F_0^t\!\left(x_0\right), t\right)\sigma\!\left(F_0^t\!\left(x_0\right), t\right)^{\T},
	                                    \end{multlined}
\end{align*}
where the second inequality has used the equation of variations \cref{eqn:eqn_of_vars}.



%%%%%%%%%%%%%%%%%%%%%%%%%%%%%%%%%%%%%%%%
\subsection{Proof of \Cref{thm:s2_calculation}}\label{app:s2_calculation_proof}
Let \(t \in [0,T]\) and consider the solutions \(y_t^{(\epsilon)}\) to \cref{eqn:sde_y} and \(l_t^{(\epsilon)}\) to \cref{eqn:linear_sde_inform} subject to the fixed initial condition \(x_0 \in \R^n\).
On the vector space of \(n\)-dimensional random vectors with each component having finite expectation and variance, define the function \(\rho\) as
\[
	\rho\!\left(z\right) \coloneqq \norm{\var{z}}^{\frac12}.
\]
Then, \(\rho\) is a semi-norm, which can be verified using properties of the spectral norm and the Cauchy-Schwarz inequality.
This proof is provided in the supplementary materials.
Then,
\begin{align*}
	\abs{\!\norm{\var{y_t^{(\epsilon)}\!}}^{\frac12}\!\!\! - \norm{\var{l_t^{(\epsilon)}\!}}^{\frac12}\!}
	 & = \abs{\rho\!\left(y_t^{(\epsilon)}\right) - \rho\!\left(l_t^{(\epsilon)}\right) }                                                                                                                                                       \\
	 & \leq \rho\!\left(y_t^{(\epsilon)} - l_t^{(\epsilon)}\right)                                                                                                                                                                              \\
	 & = \norm{\avg{\!\left(y_t^{(\epsilon)}\!- l_t^{(\epsilon)}\right)\!\left(y_t^{(\epsilon)}\!- l_t^{(\epsilon)}\right)^{\T}}\! - \!\avg{y_t^{(\epsilon)}\!- l_t^{(\epsilon)}}\!\avg{y_t^{(\epsilon)}\!- l_t^{(\epsilon)}}^{\T}\!}^{\frac12} \\
	 & \leq \left(\avg{\norm{y_t^{(\epsilon)} - l_t^{(\epsilon)}}^{2}} + \avg{\norm{y_t^{(\epsilon)} - l_t^{(\epsilon)}}}^2\right)^{\frac12}                                                                                                    \\
	 & \leq \left(\left(K_{\nabla\nabla u} + K_{\nabla\sigma}\right)D_1(2,t)+ \left(K_{\nabla\nabla u} + K_{\nabla\sigma}\right)^2 D_1(1,t)^2\right)^{1/2}\epsilon^2
\end{align*}
where the first inequality uses the reverse triangle inequality, the second inequality uses the Jensen's inequality and properties of the spectral norm, and the third inequality results from \Cref{thm:main}.
Thus,
\[
	\abs{\norm{\frac{1}{\epsilon^2}\var{y_t^{(\epsilon)}}}^{\frac12} - \norm{\frac{1}{\epsilon^2}\var{l_t^{(\epsilon)}}\!}^{\frac{1}{2}}} \leq \begin{multlined}[t]
		\left(\left(K_{\nabla\nabla u} + K_{\nabla\sigma}\right)D_1(2,t) \right. \\
		\left. + \left(K_{\nabla\nabla u} + K_{\nabla\sigma}\right)^2 D_1(1,t)^2\right)^{\frac{1}{2}}\epsilon,
	\end{multlined}
\]
and so taking the limit of \(\epsilon\) to zero and squaring both sides,
\begin{equation}\label{eqn:sigma_lim}
	\lim_{\epsilon\downarrow 0}\norm{\frac{1}{\epsilon^2}\var{y_t^{(\epsilon)}}} =
	\lim_{\epsilon\downarrow 0}\norm{\frac{1}{\epsilon^2}\var{l_t^{(\epsilon)}}} .
\end{equation}
Now, for \(\epsilon > 0\), define
\begin{align*}
	S^2_{(\epsilon)}(x_0,t) & \coloneqq \sup{\setc{\var{\frac{1}{\epsilon} p^{\T}\left(y_t^{(\epsilon)} - F_0^t\!\left(x_0\right)\right)}}{p \in \R^n, \, \norm{p} = 1}} \\
	                        & = \frac{1}{\epsilon^2}\sup{\setc{p^{\T}\var{y_t^{(\epsilon)}}p}{p \in \R^n, \, \norm{p} = 1}}
\end{align*}
Since \(\var{y_t^{(\epsilon)}}\) is symmetric and positive definite, the Cholesky decomposition provides a lower triangular \(n \times n\) matrix \(\Pi^{(\epsilon)}\) such that \(\var{y_t^{(\epsilon)}} = \Pi^{(\epsilon)}\left[\Pi^{(\epsilon)}\right]^{\T}\), allowing us to write
\begin{align*}
	S^2_{(\epsilon)}(x_0,t) & = \frac{1}{\epsilon^2}\sup\setc{\norm{\Pi^{(\epsilon)}p}^2}{p \in \R^n, \, \norm{p} = 1} \\
	                        & = \frac{1}{\epsilon^2}\norm{\Pi^{(\epsilon)}}^2                                          \\
	                        & = \frac{1}{\epsilon^2}\norm{\var{y_t^{(\epsilon)}(x)}},
\end{align*}
using properties of the spectral norm.
Taking the limit as \(\epsilon\) approaches zero and using \cref{eqn:sigma_lim},
\[
	S^2(x_0,t) = \lim_{\epsilon\downarrow 0} S^2_{(\epsilon)}(x_0,t) = \lim_{\epsilon\downarrow 0}\norm{\frac{1}{\epsilon^2}\var{y_t^{(\epsilon)}}} = \norm{\Sigma_0^t\!\left(x_0\right)},
\]
where \( \Sigma_0^t \) is defined in \Cref{eqn:sigma_def}.
Since \(\Sigma_0^t\!\left(x_0\right)\) is symmetric and positive definite, the operator norm, and therefore \(S^2\!\left(x_0,t\right)\), is given by the largest eigenvalue of \(\Sigma_0^t\!\left(x_0\right)\).
