\chapter{A Gaussian mixture model}\label{ch:gmm}


First, we shall make some adjustments to the theory as presented in \Cref{ch:limit_paper}, by dropping the explicit \(\epsilon\) notation and extending the theory to allow for Gaussian initial conditions to our stochastic differential equation.


\section{The deterministic model versus the stochastic model}

In \Cref{ch:limit_paper}, we provided a rigorous justification that the Gaussian density described in \Cref{thm:gauss_dist} provides an approximation/characterisation of the solution to a stochastic differential equation, in the sense of a small-noise limit.
The scale of the noise was explicitly parameterised with a non-zero value \(\epsilon\), and we considered the behaviour of solutions in the limit as \(\epsilon\) approaches zero.
However, in practice there will be a prescribed value of \(\epsilon\), either chosen judiciously from context or informed by data.
Henceforth, we shall drop the use of \(\epsilon\) and instead consider stochastic differential equations of the form
\[
	\sde{y_t}{u\left(y_t, t\right)}{\sigma\left(y_t, t\right)},
\]
where, strictly speaking, the noise scale parameter has been included in the diffusion term \(\sigma\).

\begin{subequations}\label{eqn:gauss_de}
	\begin{align}
		\dod{F_s^t(x)}{t}                & = u\left(F_s^t(x), t\right), \quad F_s^s(x) = x \label{eqn:gauss_mean_de}                                                              \\
		\dod{\Sigma_s^t(x; \Sigma_0)}{t} & = \begin{multlined}[t]
			                                     \nabla u\left(F_s^t(x), t\right) \Sigma_s^t(x; \Sigma_0) + \Sigma_s^t(x; \Sigma_0)\left[\nabla u\left(F_s^t(x), t\right)\right]^{\T} \\
			                                     + \sigma\left(F_s^t(x), t \right)\sigma\left(F_s^t(x), t\right)^{\T}, \quad \Sigma_s^s(x; \Sigma_0) = \Sigma_0,
		                                     \end{multlined}
		\label{eqn:gauss_cov_de}
	\end{align}
\end{subequations}



It is also worth noting that the small noise limit can be equivalently thought of, at least heuristically, as a small time limit, using scaling properties of the Wiener process.
\td{Show this or otherwise work it out. Just needs to be a heuristic or intuitive idea, rather than anything \emph{too} precise}



\section{Motivation: The Gulf Stream}

A key advantage of the Gaussian limit is the ease of computation; rather than having to generate a large number of realisations of the SDE solution to understand, either qualitatively or for the purposes of inference and estimation, the probability distribution of the solution, we can solve a smaller system of equations \eqref{eqn:sigma_ode} for the state and covariance simultaneously.

We shall first motivate the importance of stochastic models




\section{Propagating uncertain initial conditions}


\section{Solving for the state and covariance}
To compute the Gaussian limit along a deterministic trajectory, we can solve the system of equations \eqref{eqn:gauss_de}, providing that the Jacobian \(\nabla u\) of the vector field is available, or can be approximated appropriately.
Since \(\Sigma_s^t\) represents a covariance matrix, it must remain symmetric and positive semi-definite when solving \eqref{eqn:gauss_de}.
However, many standard numerical schemes do not take this into account, so a specialised scheme is required, as described below.

Similar equations of the form \eqref{eqn:gauss_de} (although often without dependence on \emph{both} time and the state in the \(\sigma\) term) are solved numerically in other applications, notably when implementing the extended Kalman filter on stochastic differential equation models \citep{Jazwinski_2014_StochasticProcessesFiltering, KulikovaKulikov_2014_AdaptiveODESolvers}.
\citet{KulikovaKulikov_2014_AdaptiveODESolvers} identify that that the two most significant sources of numerical error when solving \eqref{eqn:gauss_de} are a) the estimate of the covariance matrix \(\Sigma_s^t\) violates the requirement of positive semi-definiteness, and b) local error propagation in the state equation without an adaptive step size.
The state equation \eqref{eqn:gauss_mean_de} is the only non-linear part of \eqref{eqn:gauss_de}, so


\citet{Mazzoni_2008_ComputationalAspectsContinuous} proposes an efficient hybrid method for solving \eqref{eqn:gauss_de} which addresses both difficulties a) and b), and takes advantage of the availability of \(\nabla u\).
This method, which we shall term the Mazzoni method, combines a Taylor-Heun approximation to solve \eqref{eqn:gauss_mean_de} for the state and a Gauss-Legendre step to solve \eqref{eqn:gauss_cov_de} for the covariance.



Throughout, we use the Mazzoni method to solve \eqref{eqn:gauss_de}

\begin{subequations}\label{eqn:mazzoni_update}
	The Taylor-Heun formula for the update of the state is then
	\begin{equation}
		F_{s}^{t + \Delta t}(x) \approx F_s^{t}(x) + \left(I - \frac{\Delta t}{2}\nabla u\left(F_s^t(x), t\right)\right)^{-1}.
		\label{eqn:mazzoni_state_update}
	\end{equation}
	The Gauss-Legendre update of the covariance is
	\begin{equation}
		\Sigma_{s}^{t + \delta t}\left(x; \Sigma_0\right) \approx M_\tau \Sigma_s^t\left(x; \Sigma_0\right) M_\tau^{\T} + \Delta t K_\tau \sigma\left(w_\tau,\, t + \frac{\Delta t}{2}\right)\sigma\left(w_\tau,\, t + \frac{\Delta t}{2}\right)^{\T} K_\tau^{\T},
		\label{eqn:mazzoni_cov_update}
	\end{equation}
	where
	\begin{align}%\label{eqn:mazzoni_cov_terms}
		w_\tau & = \frac12\left(w_t + w_{t + \Delta t} - \frac{\Delta t^2}{4}\nabla u\left(w_t, \, t\right) u\left(w_t, \, t\right)\right) \\
		K_\tau & = \left[I - \frac{\Delta t}{2}\nabla u\left(w_\tau,\, t + \frac{\Delta t}{2}\right)\right]^{-1}                           \\
		M_\tau & = K_\tau \left[I + \frac{\Delta t}{2}\nabla u\left(w_\tau,\, t + \frac{\Delta t}{2}\right)\right].
	\end{align}
\end{subequations}




\section{The GMM algorithm}






\section{Analysis through exact SDE solutions}



\subsection{A linear SDE}
Consider an \(n\)-dimensional linear stochastic differential equation with additive noise;
\begin{equation}
	\sde{y_t}{A(t)y_t}{B(t)},
	\label{eqn:linear_n_sde}
\end{equation}
where \(A: [0,T] \to \R^{n\times n}\) and \(B: [0,T] \to \R^{n \times m}\) are specified, deterministic matrix-valued functions that are sufficiently smooth and measurable to ensure the existence of solutions, and \(W_t\) is an \(m\)-dimensional Wiener process.



The Gaussian limit described in \Cref{thm:gauss_dist}





\subsection{Ben\^e's SDE}





% \section{StochasticSensitivity.jl}
